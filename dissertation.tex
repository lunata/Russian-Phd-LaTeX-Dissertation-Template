%&preformat-disser
\RequirePackage[l2tabu,orthodox]{nag} % Раскомментировав, можно в логе получать рекомендации относительно правильного использования пакетов и предупреждения об устаревших и нерекомендуемых пакетах
% Формат А4, 14pt (ГОСТ Р 7.0.11-2011, 5.3.6)
\documentclass[a4paper,14pt,oneside,openany]{memoir}

\input{common/setup}            % общие настройки шаблона
\input{common/packages}         % Пакеты общие для диссертации и автореферата
\synopsisfalse                      % Этот документ --- не автореферат
\input{Dissertation/dispackages}    % Пакеты для диссертации
\input{Dissertation/userpackages}   % Пакеты для специфических пользовательских задач

\input{Dissertation/setup}      % Упрощённые настройки шаблона

\input{common/newnames}         % Новые переменные, для всего проекта

%%% Основные сведения %%%
\newcommand{\thesisAuthorLastName}{Крижановская}
\newcommand{\thesisAuthorOtherNames}{Наталья Борисовна}
\newcommand{\thesisAuthorInitials}{Н.\,Б.}
\newcommand{\thesisAuthor}             % Диссертация, ФИО автора
{%
    \texorpdfstring{% \texorpdfstring takes two arguments and uses the first for (La)TeX and the second for pdf
        \thesisAuthorLastName~\thesisAuthorOtherNames% так будет отображаться на титульном листе или в тексте, где будет использоваться переменная
    }{%
        \thesisAuthorLastName, \thesisAuthorOtherNames% эта запись для свойств pdf-файла. В таком виде, если pdf будет обработан программами для сбора библиографических сведений, будет правильно представлена фамилия.
    }
}
\newcommand{\thesisAuthorShort}        % Диссертация, ФИО автора инициалами
{\thesisAuthorInitials~\thesisAuthorLastName}
%\newcommand{\thesisUdk}                % Диссертация, УДК
%{\todo{xxx.xxx}}
\newcommand{\thesisTitle}              % Диссертация, название
{Методы и алгоритмы построения компьютерной модели морфологического
словаря на примере именного словоизменения вепсского языка}
\newcommand{\thesisSpecialtyNumber}    % Диссертация, специальность, номер
{05.13.18}
\newcommand{\thesisSpecialtyTitle}     % Диссертация, специальность, название
{Математическое моделирование, численные методы и комплексы программ}
\newcommand{\thesisDegree}             % Диссертация, ученая степень
{кандидата технических наук}
\newcommand{\thesisDegreeShort}        % Диссертация, ученая степень, краткая запись
{канд. тех. наук}
\newcommand{\thesisCity}               % Диссертация, город написания диссертации
{Петрозаводск}
\newcommand{\thesisYear}               % Диссертация, год написания диссертации
{\todo{20XX}}
\newcommand{\thesisOrganization}       % Диссертация, организация
{Институт прикладных математических исследований КарНЦ РАН}
\newcommand{\thesisOrganizationShort}  % Диссертация, краткое название организации для доклада
{\todo{НазУчДисРаб}}

\newcommand{\thesisInOrganization}     % Диссертация, организация в предложном падеже: Работа выполнена в ...
{Институте прикладных математических исследований}

\newcommand{\supervisorFio}            % Научный руководитель, ФИО
{Крижановский Андрей Анатольевич}
\newcommand{\supervisorRegalia}        % Научный руководитель, регалии
{к.т.н.}
\newcommand{\supervisorFioShort}       % Научный руководитель, ФИО
{А.\,А.~Крижановский}
\newcommand{\supervisorRegaliaShort}   % Научный руководитель, регалии
{к.т.н.}


\newcommand{\opponentOneFio}           % Оппонент 1, ФИО
{\todo{Фамилия Имя Отчество}}
\newcommand{\opponentOneRegalia}       % Оппонент 1, регалии
{\todo{доктор физико-математических наук, профессор}}
\newcommand{\opponentOneJobPlace}      % Оппонент 1, место работы
{\todo{Не очень длинное название для места работы}}
\newcommand{\opponentOneJobPost}       % Оппонент 1, должность
{\todo{старший научный сотрудник}}

\newcommand{\opponentTwoFio}           % Оппонент 2, ФИО
{\todo{Фамилия Имя Отчество}}
\newcommand{\opponentTwoRegalia}       % Оппонент 2, регалии
{\todo{кандидат физико-математических наук}}
\newcommand{\opponentTwoJobPlace}      % Оппонент 2, место работы
{\todo{Основное место работы c длинным длинным длинным длинным названием}}
\newcommand{\opponentTwoJobPost}       % Оппонент 2, должность
{\todo{старший научный сотрудник}}

\newcommand{\leadingOrganizationTitle} % Ведущая организация, дополнительные строки
{\todo{Федеральное государственное бюджетное образовательное учреждение высшего профессионального образования с~длинным длинным длинным длинным названием}}

\newcommand{\defenseDate}              % Защита, дата
{\todo{DD mmmmmmmm YYYY~г.~в~XX часов}}
\newcommand{\defenseCouncilNumber}     % Защита, номер диссертационного совета
{\todo{Д\,123.456.78}}
\newcommand{\defenseCouncilTitle}      % Защита, учреждение диссертационного совета
{\todo{Название учреждения}}
\newcommand{\defenseCouncilAddress}    % Защита, адрес учреждение диссертационного совета
{\todo{Адрес}}
\newcommand{\defenseCouncilPhone}      % Телефон для справок
{\todo{+7~(0000)~00-00-00}}

\newcommand{\defenseSecretaryFio}      % Секретарь диссертационного совета, ФИО
{\todo{Фамилия Имя Отчество}}
\newcommand{\defenseSecretaryRegalia}  % Секретарь диссертационного совета, регалии
{\todo{д-р~физ.-мат. наук}}            % Для сокращений есть ГОСТы, например: ГОСТ Р 7.0.12-2011 + http://base.garant.ru/179724/#block_30000

\newcommand{\synopsisLibrary}          % Автореферат, название библиотеки
{\todo{Название библиотеки}}
\newcommand{\synopsisDate}             % Автореферат, дата рассылки
{\todo{DD mmmmmmmm YYYY года}}

% To avoid conflict with beamer class use \providecommand
\providecommand{\keywords}%            % Ключевые слова для метаданных PDF диссертации и автореферата
{}
             % Основные сведения
\input{common/fonts}            % Определение шрифтов (частичное)
\input{common/styles}           % Стили общие для диссертации и автореферата
\input{Dissertation/disstyles}  % Стили для диссертации
\input{Dissertation/userstyles} % Стили для специфических пользовательских задач

% жирная черта слева (или справа) от утверждения (таблица)
% https://tex.stackexchange.com/a/256732/99685
%\newcommand\thickPipe{!{\vrule width 4pt}}
% arguments: text and width in cm
\newcommand\textWithVerticalBar[2]{%
   \begin{tabular}{!{\vrule width 4pt}p{#1}}%
    #2%
    \end{tabular}%
}
 % Мои стили

%%% Библиография. Выбор движка для реализации %%%
% Здесь только проверка установленного ключа. Сама настройка выбора движка
% размещена в common/setup.tex
\ifnumequal{\value{bibliosel}}{0}{%
    \input{biblio/predefined}   % Встроенная реализация с загрузкой файла через движок bibtex8
}{
    \input{biblio/biblatex}     % Реализация пакетом biblatex через движок biber
}

% Вывести информацию о выбранных опциях в лог сборки
\typeout{Selected options:}
\typeout{Draft mode: \arabic{draft}}
\typeout{Font: \arabic{fontfamily}}
\typeout{AltFont: \arabic{usealtfont}}
\typeout{Bibliography backend: \arabic{bibliosel}}
\typeout{Precompile images: \arabic{imgprecompile}}
% Вывести информацию о версиях используемых библиотек в лог сборки
\listfiles

%%% Управление компиляцией отдельных частей диссертации %%%
% Необходимо сначала иметь полностью скомпилированный документ, чтобы все
% промежуточные файлы были в наличии
% Затем, для вывода отдельных частей можно воспользоваться командой \includeonly
% Ниже примеры использования команды:
%
%\includeonly{Dissertation/part2}
%\includeonly{Dissertation/contents,Dissertation/appendix,Dissertation/conclusion}
%
% Если все команды закомментированы, то документ будет выведен в PDF файл полностью

\hyphenation{вепс-ский}

\begin{document}

\input{common/renames}                 % Переопределение именований

%%% Структура диссертации (ГОСТ Р 7.0.11-2011, 4)
\include{Dissertation/title}           % Титульный лист
\include{Dissertation/contents}        % Оглавление

\ifnumequal{\value{contnumfig}}{1}{}{\counterwithout{figure}{chapter}}
\ifnumequal{\value{contnumtab}}{1}{}{\counterwithout{table}{chapter}}

\chapter*{Введение}                         % Заголовок
\addcontentsline{toc}{chapter}{Введение}    % Добавляем его в оглавление

\newcommand{\actuality}{}
\newcommand{\progress}{}
\newcommand{\aim}{{\textbf\aimTXT}}
\newcommand{\tasks}{\textbf{\tasksTXT}}
\newcommand{\novelty}{\textbf{\noveltyTXT}}
\newcommand{\influence}{\textbf{\influenceTXT}}
\newcommand{\methods}{\textbf{\methodsTXT}}
\newcommand{\defpositions}{\textbf{\defpositionsTXT}}
\newcommand{\reliability}{\textbf{\reliabilityTXT}}
\newcommand{\probation}{\textbf{\probationTXT}}
\newcommand{\contribution}{\textbf{\contributionTXT}}
\newcommand{\publications}{\textbf{\publicationsTXT}}


{\actuality} Обзор, введение в тему, обозначение места данной работы в
мировых исследованиях и~т.\:п., можно использовать ссылки на~другие
работы~\autocite{Kibrik2019}
(если их~нет, то~в~автореферате
автоматически пропадёт раздел <<Список литературы>>). Внимание! Ссылки
на~другие работы в разделе общей характеристики работы можно
использовать только при использовании \verb!biblatex! (из-за технических
ограничений \verb!bibtex8!. Это связано с тем, что одна
и~та~же~характеристика используются и~в~тексте диссертации, и в
автореферате. В~последнем, согласно ГОСТ, должен присутствовать список
работ автора по~теме диссертации, а~\verb!bibtex8! не~умеет выводить в одном
файле два списка литературы).
При использовании \verb!biblatex! возможно использование исключительно
в~автореферате подстрочных ссылок
для других работ командой \verb!\autocite!, а~также цитирование
собственных работ командой \verb!\cite!. Для этого в~файле
\verb!common/setup.tex! необходимо присвоить положительное значение
счётчику \verb!\setcounter{usefootcite}{1}!.

Для генерации содержимого титульного листа автореферата, диссертации
и~презентации используются данные из файла \verb!common/data.tex!. Если,
например, вы меняете название диссертации, то оно автоматически
появится в~итоговых файлах после очередного запуска \LaTeX. Согласно
ГОСТ 7.0.11-2011 <<5.1.1 Титульный лист является первой страницей
диссертации, служит источником информации, необходимой для обработки и
поиска документа>>. Наличие логотипа организации на~титульном листе
упрощает обработку и~поиск, для этого разметите логотип вашей
организации в папке images в~формате PDF (лучше найти его в векторном
варианте, чтобы он хорошо смотрелся при печати) под именем
\verb!logo.pdf!. Настроить размер изображения с логотипом можно
в~соответствующих местах файлов \verb!title.tex!  отдельно для
диссертации и автореферата. Если вам логотип не~нужен, то просто
удалите файл с~логотипом.

\ifsynopsis
Этот абзац появляется только в~автореферате.
Для формирования блоков, которые будут обрабатываться только в~автореферате,
заведена проверка условия \verb!\!\verb!ifsynopsis!.
Значение условия задаётся в~основном файле документа (\verb!synopsis.tex! для
автореферата).
\else
Этот абзац появляется только в~диссертации.
Через проверку условия \verb!\!\verb!ifsynopsis!, задаваемого в~основном файле
документа (\verb!dissertation.tex! для диссертации), можно сделать новую
команду, обеспечивающую появление цитаты в~диссертации, но~не~в~автореферате.
\fi

% {\progress} 
% Этот раздел должен быть отдельным структурным элементом по
% ГОСТ, но он, как правило, включается в описание актуальности
% темы. Нужен он отдельным структурынм элемементом или нет ---
% смотрите другие диссертации вашего совета, скорее всего не нужен.

{\aim} данной работы является разработка методов и алгоритмов построения компьютерной модели морфологического словаря с целью решения задачи лемматизации на примере именного словоизменения карельского языка. 

Для~достижения поставленной цели необходимо было решить следующие {\tasks}:
\begin{enumerate}[beginpenalty=10000] % https://tex.stackexchange.com/a/476052/104425
  \item Исследовать, разработать, вычислить и~т.\:д. и~т.\:п.
  \item Исследовать, разработать, вычислить и~т.\:д. и~т.\:п.
  \item Исследовать, разработать, вычислить и~т.\:д. и~т.\:п.
  \item Исследовать, разработать, вычислить и~т.\:д. и~т.\:п.
\end{enumerate}


{\novelty}
\begin{enumerate}[beginpenalty=10000] % https://tex.stackexchange.com/a/476052/104425
  \item Впервые \ldots
  \item Впервые \ldots
  \item Было выполнено оригинальное исследование \ldots
\end{enumerate}

{\influence}. На важность поставленной проблемы указывает тот факт, что любая задача, связанная с обработкой текста, требует первичной обработки текста, которая включает лемматизацию, то есть приведение слов к начальной форме.

Для успешного развития машинных методов обработки языка жизненно необходимо наличие трех компонент: корпуса текстов, словаря и морфологического анализатора (лемматизатора). Для работы с текстами на русском языке есть корпуса (Национальный корпус русского языка, OpenCorpora), есть словари и лемматизаторы (MyStem, pymorphy2). Для карельского языка есть корпус и словарь ВепКар, но лемматизатора нет.

Для обработки текстов на карельском языке, для машинного перевода текстов с карельского и на карельский язык нужен морфологический анализатор. Таким образом, задача построения такого анализатора будет не только своевременна (поскольку такой анализатор требуется в Открытом корпусе вепсского и карельского языков), но и будет серьезным научным достижением для прибалтийско-финского языкознания. На данный момент нет сведений о существовании доступного лемматизатора карельского языка \todo{Todo}.

Укажу на взаимосвязь задачи построения морфологического анализатора с другими задачами вычислительной лингвистики. То есть определю связанные задачи, которые покажут комплексность поставленной проблемы. Итак, в ходе исследовательской работы потребуется решение следующих задач:
\begin{itemize}
\item сегментация и нормализация текста,
\item определение частеречной принадлежности (POS tagging),
\item определение морфологических свойств слова,
\item определение леммы слова для словарных и для неизвестных слов,
\item создание тестовой коллекции для оценки работы морфологических парсеров карельского языка.
\end{itemize}


{\methods} В работе используются 
методы создания и разметки (аннотирования) лингвистических корпусов~\cite[с.~408]{Kibrik2019}, 
методы морфологического анализа 
(трансдьюсеры на основе конечных автоматов, 
преобразования на основе правил, 
ещё..?). 

Используются технологии поиска в корпусе текстов, 
\textcolor{magenta}{разрабатывается специализированный язык запрос к корпусам}~\cite[с.~410]{Kibrik2019}.
\todo{Todo: если будет разработан язык запросов, то это добавить в положение на защиту. 
Тогда нужен будет обзор языков.}

{\defpositions}
\begin{enumerate}[beginpenalty=10000] % https://tex.stackexchange.com/a/476052/104425
  \item Первое положение
  \item Второе положение
  \item Третье положение
  \item Четвертое положение
\end{enumerate}
В папке Documents можно ознакомиться в решением совета из Томского ГУ
в~файле \verb+Def_positions.pdf+, где обоснованно даются рекомендации
по~формулировкам защищаемых положений. 

{\reliability} Достоверность научных положений, основных выводов и результатов диссертационной работы 
основывается на анализе работ в этой области, согласованности предложенных моделей 
\ldots \todo{(какие модели и алгоритмы были разработаны, перечислить)} \ldots 
с результатами экспериментов, полученных на основе разработанного комплекса компьютерных программ. 
Также достоверность подтверждается апробацией основных положений диссертации в печатных трудах 
и докладах на научных конференциях. 
Новизна технических решений подтверждается полученными свидетельствами на программы для ЭВМ. 
Результаты работы представлялись на международной конференции <<Корпусная лингвистика>> в 2017, 2019 гг.; 
международной конференции <<Диалог>> в 2019 г.;
международной конференции <<Электронная письменность народов Российской Федерации: опыт, проблемы и перспективы>> в 2019 г., 
\ldots
\ldots; всероссийской конференции <<>> в 2019 г., \ldots 
Получены свидетельства о регистрации программы для ЭВМ \textnumero~2019665163 от 20.11.2019 "New written Tver Karelian dialects wordform generator" и 
\ldots
(см. Приложение А.).



{\probation}
Основные результаты работы докладывались~на:
перечисление основных конференций, симпозиумов и~т.\:п.

{\contribution} Автор принимал активное участие \ldots

\ifnumequal{\value{bibliosel}}{0}
{%%% Встроенная реализация с загрузкой файла через движок bibtex8. (При желании, внутри можно использовать обычные ссылки, наподобие `\cite{vakbib1,vakbib2}`).
    {\publications} Основные результаты по теме диссертации изложены
    в~XX~печатных изданиях,
    X из которых изданы в журналах, рекомендованных ВАК,
    X "--- в тезисах докладов.
}%
{%%% Реализация пакетом biblatex через движок biber
    \begin{refsection}[bl-author, bl-registered]
        % Это refsection=1.
        % Процитированные здесь работы:
        %  * подсчитываются, для автоматического составления фразы "Основные результаты ..."
        %  * попадают в авторскую библиографию, при usefootcite==0 и стиле `\insertbiblioauthor` или `\insertbiblioauthorgrouped`
        %  * нумеруются там в зависимости от порядка команд `\printbibliography` в этом разделе.
        %  * при использовании `\insertbiblioauthorgrouped`, порядок команд `\printbibliography` в нём должен быть тем же (см. biblio/biblatex.tex)
        %
        % Невидимый библиографический список для подсчёта количества публикаций:
        \ifxetexorluatex\selectlanguage{english}\fi
        \printbibliography[heading=nobibheading, section=1, env=countauthorvak,          keyword=biblioauthorvak]%
        \printbibliography[heading=nobibheading, section=1, env=countauthorwos,          keyword=biblioauthorwos]%
        \printbibliography[heading=nobibheading, section=1, env=countauthorscopus,       keyword=biblioauthorscopus]%
        \printbibliography[heading=nobibheading, section=1, env=countauthorconf,         keyword=biblioauthorconf]%
        \printbibliography[heading=nobibheading, section=1, env=countauthorother,        keyword=biblioauthorother]%
        \printbibliography[heading=nobibheading, section=1, env=countregistered,         keyword=biblioregistered]%
        \printbibliography[heading=nobibheading, section=1, env=countauthorpatent,       keyword=biblioauthorpatent]%
        \printbibliography[heading=nobibheading, section=1, env=countauthorprogram,      keyword=biblioauthorprogram]%
        \printbibliography[heading=nobibheading, section=1, env=countauthor,             keyword=biblioauthor]%
        \printbibliography[heading=nobibheading, section=1, env=countauthorvakscopuswos, filter=vakscopuswos]%
        \printbibliography[heading=nobibheading, section=1, env=countauthorscopuswos,    filter=scopuswos]%
        %
        \nocite{*}\ifxetexorluatex\selectlanguage{russian}\fi%
        %
        {\publications} Основные результаты по теме диссертации изложены в~\arabic{citeauthor}~печатных изданиях,
        \arabic{citeauthorvak} из которых изданы в журналах, рекомендованных ВАК\sloppy%
        \ifnum \value{citeauthorscopuswos}>0%
            , \arabic{citeauthorscopuswos} "--- в~периодических научных журналах, индексируемых Web of~Science и Scopus\sloppy%
        \fi%
        \ifnum \value{citeauthorconf}>0%
            , \arabic{citeauthorconf} "--- в~тезисах докладов.
        \else%
            .
        \fi%
        \ifnum \value{citeregistered}=1%
            \ifnum \value{citeauthorpatent}=1%
                Зарегистрирован \arabic{citeauthorpatent} патент.
            \fi%
            \ifnum \value{citeauthorprogram}=1%
                Зарегистрирована \arabic{citeauthorprogram} программа для ЭВМ.
            \fi%
        \fi%
        \ifnum \value{citeregistered}>1%
            Зарегистрированы\ %
            \ifnum \value{citeauthorpatent}>0%
            \formbytotal{citeauthorpatent}{патент}{}{а}{}\sloppy%
            \ifnum \value{citeauthorprogram}=0 . \else \ и \fi%
            \fi%
            \ifnum \value{citeauthorprogram}>0%
            \formbytotal{citeauthorprogram}{программ}{а}{ы}{} для ЭВМ.
            \fi%
        \fi%
        % К публикациям, в которых излагаются основные научные результаты диссертации на соискание учёной
        % степени, в рецензируемых изданиях приравниваются патенты на изобретения, патенты (свидетельства) на
        % полезную модель, патенты на промышленный образец, патенты на селекционные достижения, свидетельства
        % на программу для электронных вычислительных машин, базу данных, топологию интегральных микросхем,
        % зарегистрированные в установленном порядке.(в ред. Постановления Правительства РФ от 21.04.2016 N 335)
    \end{refsection}%
    \begin{refsection}[bl-author, bl-registered]
        % Это refsection=2.
        % Процитированные здесь работы:
        %  * попадают в авторскую библиографию, при usefootcite==0 и стиле `\insertbiblioauthorimportant`.
        %  * ни на что не влияют в противном случае
        \nocite{vakbib2}%vak
        \nocite{patbib1}%patent
        \nocite{progbib1}%program
        \nocite{bib1}%other
        \nocite{confbib1}%conf
    \end{refsection}%
        %
        % Всё, что вне этих двух refsection, это refsection=0,
        %  * для диссертации - это нормальные ссылки, попадающие в обычную библиографию
        %  * для автореферата:
        %     * при usefootcite==0, ссылка корректно сработает только для источника из `external.bib`. Для своих работ --- напечатает "[0]" (и даже Warning не вылезет).
        %     * при usefootcite==1, ссылка сработает нормально. В авторской библиографии будут только процитированные в refsection=0 работы.
        %
        % Невидимый библиографический список для подсчёта количества внешних публикаций
        % Используется, чтобы убрать приставку "А" у работ автора, если в автореферате нет
        % цитирований внешних источников.
        % Замедляет компиляцию
    \ifsynopsis
    \ifnumequal{\value{draft}}{0}{
      \printbibliography[heading=nobibheading, section=0, env=countexternal,          keyword=biblioexternal]%
    }{}
    \fi
}

При использовании пакета \verb!biblatex! будут подсчитаны все работы, добавленные
в файл \verb!biblio/author.bib!. Для правильного подсчёта работ в~различных
системах цитирования требуется использовать поля:
\begin{itemize}
        \item \texttt{authorvak} если публикация индексирована ВАК,
        \item \texttt{authorscopus} если публикация индексирована Scopus,
        \item \texttt{authorwos} если публикация индексирована Web of Science,
        \item \texttt{authorconf} для докладов конференций,
        \item \texttt{authorpatent} для патентов,
        \item \texttt{authorprogram} для зарегистрированных программ для ЭВМ,
        \item \texttt{authorother} для других публикаций.
\end{itemize}
Для подсчёта используются счётчики:
\begin{itemize}
        \item \texttt{citeauthorvak} для работ, индексируемых ВАК,
        \item \texttt{citeauthorscopus} для работ, индексируемых Scopus,
        \item \texttt{citeauthorwos} для работ, индексируемых Web of Science,
        \item \texttt{citeauthorvakscopuswos} для работ, индексируемых одной из трёх баз,
        \item \texttt{citeauthorscopuswos} для работ, индексируемых Scopus или Web of~Science,
        \item \texttt{citeauthorconf} для докладов на конференциях,
        \item \texttt{citeauthorother} для остальных работ,
        \item \texttt{citeauthorpatent} для патентов,
        \item \texttt{citeauthorprogram} для зарегистрированных программ для ЭВМ,
        \item \texttt{citeauthor} для суммарного количества работ.
\end{itemize}
% Счётчик \texttt{citeexternal} используется для подсчёта процитированных публикаций;
% \texttt{citeregistered} "--- для подсчёта суммарного количества патентов и программ для ЭВМ.

Для добавления в список публикаций автора работ, которые не были процитированы в
автореферате, требуется их~перечислить с использованием команды \verb!\nocite! в
\verb!Synopsis/content.tex!.
 % Характеристика работы по структуре во введении и в автореферате не отличается (ГОСТ Р 7.0.11, пункты 5.3.1 и 9.2.1), потому её загружаем из одного и того же внешнего файла, предварительно задав форму выделения некоторым параметрам

\textbf{Объем и структура работы.} Диссертация состоит из~введения, трёх глав,
заключения и~двух приложений.
%% на случай ошибок оставляю исходный кусок на месте, закомментированным
%Полный объём диссертации составляет  \ref*{TotPages}~страницу
%с~\totalfigures{}~рисунками и~\totaltables{}~таблицами. Список литературы
%содержит \total{citenum}~наименований.
%
Полный объём диссертации составляет
\formbytotal{TotPages}{страниц}{у}{ы}{}, включая
\formbytotal{totalcount@figure}{рисун}{ок}{ка}{ков} и
\formbytotal{totalcount@table}{таблиц}{у}{ы}{}.   Список литературы содержит
\formbytotal{citenum}{наименован}{ие}{ия}{ий}.

Разбить по частям и главам текст:

Цель исследования -- разработка методов и алгоритмов построения компьютерной модели морфологического словаря с целью решения задачи лемматизации на примере карельского языка. Для упрощения работы будет рассматриваться только именное словоизменение, то есть я ограничусь карельскими существительными, прилагательными, числительными и местоимениями.
На важность поставленной проблемы указывает тот факт, что любая задача, связанная с обработкой текста, требует первичной обработки текста, которая включает лемматизацию, то есть приведение слов к начальной форме.

Для успешного развития машинных методов обработки языка жизненно необходимо наличие трех компонент: корпуса текстов, словаря и морфологического анализатора (лемматизатора). Для работы с текстами на русском языке есть корпуса (Национальный корпус русского языка, OpenCorpora), есть словари и лемматизаторы (MyStem, pymorphy2). Для карельского языка есть корпус и словарь ВепКар, но лемматизатора нет.

Для обработки текстов на карельском языке, для машинного перевода текстов с карельского и на карельский язык нужен морфологический анализатор. Таким образом, задача построения такого анализатора будет не только своевременна (поскольку такой анализатор требуется в Открытом корпусе вепсского и карельского языков), но и будет серьезным научным достижением для прибалтийско-финского языкознания. На данный момент нет сведений о существовании доступного лемматизатора карельского языка \todo.

Укажу на взаимосвязь задачи построения морфологического анализатора с другими задачами вычислительной лингвистики. То есть определю связанные задачи, которые покажут комплексность поставленной проблемы. Итак, в ходе исследовательской работы потребуется решение следующих задач:
\begin{itemize}
\item сегментация и нормализация текста,
\item определение частеречной принадлежности (POS tagging),
\item определение морфологических свойств слова,
\item определение леммы слова для словарных и для неизвестных слов,
\item создание тестовой коллекции для оценки работы морфологических парсеров карельского языка.
\end{itemize}
    % Введение
%
%%%%%%%%%%%%%%%%%%%%%%%%%%%%%%%%%%%%%%%%%%%%%%%%%
%
% Введение в главу "Обзор"
\chapter{Обзор} \label{chapt_review}

\TODO{TODO: Написать вводную для обзорной главы с перечислением всего обозреваемого.}


\emph{Заметки на полях: раскидать по кирпичу}

The CoNLL--SIGMORPHON 2018 \aka{Пример примечания не по делу. Высмотрел в SIGMORPHON 2020.}  Shared Task: Universal Morphological Reinflection

https://arxiv.org/pdf/1810.07125.pdf

Для 103 языков были даны леммы и морфологические характеристики, 
нужно было получить правильную словоформу -- это первая задача. 
Вторая -- дан фрагмент текста, нужно для слова указать его морфологические 
характеристики (и лемму?).

Разработано программное обеспечение wcorpus~\cite{vakbib_soft_wcorpus}.

Разработана программа ``New written Tver Karelian dialects wordform generator''~\cite{vakbib_soft_Tver_generator}.

\todo[color=green!40]{And a green note}





     
% 
% Корпусная лингвистика
\section{Корпусная лингвистика}\label{sect_review_corpus_linguistics}

Корпусная лингвистика~--- это раздел компьютерной (прикладной) лингвистики, 
описывающий принципы и методы 
построения лингвистических корпусов (корпусов текстов)
и методы использования корпусных данных~\cite[3]{Zakharov2005},~\cite[407]{Kibrik2019}.


Корпус можно рассматривать как \emph{метод поиска} 
на больших массивах текстов~\cite[18]{Kozera2019CorpusAsMethod}. 
С помощью корпусов можно оценить частотность языковых конструкций, 
выявить образцы сочетаемости слов~\cite[18]{Kozera2019CorpusAsMethod}. 


Научное значение корпусов заключается в том, что наличие корпуса обеспечивает воспроизводимость, 
возможность повторить эксперимент~\cite[409]{Kibrik2019} 
хотя бы в пределах одного корпуса\footnote{%
    Рассчитывать на воспроизводимость эксперимента в разных корпусах можно 
    при соблюдении требования \emph{репрезентативности}, см. следующий раздел.
}. 
Трудность здесь может крыться в том, что <<живые>> корпусы, 
то есть те, над которыми продолжают работать исследователи, 
постоянно пополняются новыми текстами, увеличивается объём разметки. 
Этот рост корпуса может менять результаты эксперимента. 
Ситуацию здесь может спасти то, 
что в достаточно больших корпусах добавление новых данных будет небольшим 
относительно объёма данных всего корпуса\footnote{%
    %
    % Объём и в процентах новых слов в ВепКар 
    Приведём обратный пример, пример о добавлении большого объёма новых данных 
    в относительно небольшой корпус ВепКар. 
    Были обработаны и преобразованы в машиночаемую форму словарные статьи 
    Сопоставительно-ономасиологического словаря диалектов карельского, вепсского, саамского языков (кратко, словарь СОСД)~\cite{SOSD2007}. 
    %одержащего 1500 понятий 
    %\TODO{TODO можно ли чуть подробнее о СОСД? Сколько типов связей и сколько связей он содержит? Сколько диалектов? Чем он грандиозен?} 
    До включения данных словаря СОСД система ВепКар содержала 35~098 лемм, 800 переводов.
    После обработки СОСД в ВепКар было включено 1425 понятий 
        (\TODO{одно понятие через значение связывает несколько десятков диалектов}), 
    20 тыс. лемм, связанных с понятиями, из них 16 тыс. новых (\TODO{то есть число лемм выросло на 46\% от 35 тыс.}); 
    130 тыс. переводов (связей между значениями лемм из разных языков, наречий). 
    При этом было создано 60 тыс. связей между значениями лемм и словами из текста, 
    50 тыс. слов в текстах получили новые связи. Но только 2 тыс. слов до этого не имели разметки вообще.
    Эти цифры говорят и о том, что Сопоставительно-ономасиологический словарь был грандиозным проектом, 
    содержащим результаты работы большого коллектива учёных, и о том,
    что корпус ВепКар ещё далёк от насыщения по числу слов и текстов.
}.


 
%
% Репрезентативность и сбалансированность корпуса, достоверность данных

\subsection{Полнота, репрезентативность корпуса, достоверность корпусных исследований} \label{sect_corpus_representativeness}

Для исследователей, работающих с корпусом, важен факт, что 
явление, не встретившиеся в корпусе, не получит отражения в описании. 
Поэтому 
<<как бы ни было редко (исследуемое явление) в языке, 
хотелось бы чтобы оно нашло отражение в корпусе>>~\cite[с.~413]{Kibrik2019}. 
Таким образом, \emph{требование полноты корпуса}~--- это включение таких текстов, 
чтобы исследуемое явление было представлено в корпусе 
(и желательно во всём своём многообразии).


\emph{Репрезентативность корпуса} 
(не путать с представительностью)\footnote{Представительный корпус -- 
это корпус, обеспечивающий максимально широкое покрытие  
различных типов текстов и функциональных стилей~\cite{Sharov2004}.
} 
(текстовых примеров) 
эмпирически определяют тем, в какой степени эти примеры показывают 
вариативность исследуемого явления~\cite{Biber1993representativeness}. 
%
\emph{Требование репрезентативности}~--- это выбор такого подмножества текстов 
для включения в корпус, 
чтобы выбранные тексты 
<<отражали те или иные параметры исследуемого языкового явления в той же пропорции, 
что и в языке вообще или в некотором исследуемом подмножестве языка>>~\cite[с.~413]{Kibrik2019}.

Из этих требований и определений следует, что 
нельзя говорить об универсальной полноте или репрезентативности корпуса, 
но можно говорить 
о степени репрезентативности корпуса при решении конкретной лингвистической задачи.



В работе~\cite{Belikov2013}
оценивается достоверность корпусных исследований. 
Часто исследователи подвержены соблазну распространить опыт, полученный на конкретном корпусе, 
на весь язык, что неправомерно~\cite{Belikov2013}.

Каковы границы применимости разрабатываемого Открытого корпуса вепсского и карельского языков? 
Для ответа на этот вопрос нужно определить, тексты каких жанров и в какой пропорции включены в корпус ВепКар. 
Объём корпуса ВепКар составляет соответственно 3~007 текстов и 1~084~679 слов в текстах\footnote{ Данные на 10 февраля 2021~г. Cм. подробнее 
\href{http://dictorpus.krc.karelia.ru/ru/stats/by\_corp}{http://dictorpus.krc.karelia.ru/ru/stats/by\_corp}.}.

На рисунках~\ref{fig:text_distr_by_corpus} и \ref{fig:text_distr_by_genre} 
показано распределение текстов по подкорпусам и жанрам для вепсского языка 
и наречий карельского языка\footnote{Данные 
                                     на 23 апреля 2020~года. Cм. подробнее 
			\href{http://dictorpus.krc.karelia.ru/ru/corpus/corpus}{http://dictorpus.krc.karelia.ru/ru/corpus/corpus}.}.

\begin{figure}
    \centering
    \includegraphics[width=1.0\textwidth,keepaspectratio=true]{text_distr_by_corpus.png}
    \caption[Распределение текстов по подкорпусам]{Распределение текстов по подкорпусам (вепсский язык и наречия карельского языка).}
    \label{fig:text_distr_by_corpus}
\end{figure}
%\bigskip

\begin{figure}
    \centering
    \includegraphics[width=1.0\textwidth,keepaspectratio=true]{text_distr_by_genre.png}
    \caption[Распределение текстов по жанрам]{Распределение текстов по жанрам (вепсский язык и наречия карельского языка).}
    \label{fig:text_distr_by_genre}
\end{figure}

На рис.~\ref{fig:text_distribution_by_date} показана гистограмма 
с числом текстов корпуса ВепКар по годам с 1918 по 2018 год. 
2041~текст (86,67\% или $6/7$ от всего числа текстов) не имеют информации о дате записи.
\begin{figure}
    \centering
    \includegraphics[width=1.0\textwidth,keepaspectratio=true]{text_distribution_by_date.png}
    \caption{Распределение числа текстов по годам.}
    \label{fig:text_distribution_by_date}
\end{figure}

Своевременно ли говорить о сбалансированности и представительности корпуса ВепКар? Вопрос остаётся открытым, нужны дополнительные исследования.
%Ответом может послужить информация о доле слов из словаря, употреблённых в текстах корпуса. 
%\nata{TODO: Подсчитать долю / процент слов (для каждого из языков), которые встречаются в текстах. Процент + абсолютное число слов словаря в текстах.}
%(Добавить эту статистику в корпус?)




 
%
% Примеры лингвистических корпусов
 \subsection{Примеры лингвистических корпусов}

Бывают разные подходы к разработке корпусов\ldots 

В мире существуют сотни больших лингвистических корпусов, например, Национальный корпус русского языка\footnote{См. \url{https://ruscorpora.ru}}, Британский национальный корпус\footnote{См. \url{http://www.natcorp.ox.ac.uk}}, 
Чешский национальный корпус\footnote{См. \url{https://www.korpus.cz}}. 

Для долговременной работы над корпусом нужен научный коллектив, включение корпусных исследований в план научной работы. Например, для работ над Чешским национальным корпусом был создан одноимённый институт, над созданием и пополнением Национального корпуса русского языка работают сотрудники нескольких университетов и институтов РАН. 

Хотя есть обратный пример веб-корпусов уральских языков, построенных практически в одиночку Т. Архангельским. Разница здесь в том, что веб-корпуса строятся с помощью автоматической обработки текстов сети Интернет и содержат только автоматическую разметку. Большие коллективы лингвистов нужны, чтобы выполнять тонкую настройку, то есть делать ручную разметку, создавать так называемый «золотой стандарт» или размеченную вручную часть корпуса, которая в дальнейшем будет использоваться в различных экспериментах. Примером может служить глубоко аннотированный корпус текстов русского языка СинТагРус, где каждое слово в тексте привязано к какой-либо словарной статье комбинаторного словаря \cite{Inshakova2019}.


\subsubsection{Британский национальный корпус}

\subsubsection{Чешский национальный корпус}

\subsubsection{Национальный корпус русского языка}

\subsubsection{СинТагРус}

 \subsubsection{Europarl Corpus and Words2Grids}

В работе~\cite{Fam2018tools} различают две структуры: \emph{Paradigm tables} 
и \emph{Analogical grids}. 

На языке Python разработана программа Words2Grids, которая по списку словоформ 
создаёт Analogical grids, которые, в свою очередь, нужны для создания 
таблиц склонений, то есть Paradigm tables.

Эксперименты проводились на 11 языках в корпусе текстов Europarl. 
Особняком стоят результаты по финскому (агглюнативному) языку (см. табл. 1 на с. 1063).


\subsubsection{Морфологическая разметка ГИКРЯ}

Читать и писать о работе~\cite{Selegey2016}...

\subsubsection{Корпуса уральских языков Поволжья}
Т. Архангельским разработаны веб-корпусы для пяти уральских языков\footnote{См. \url{http://volgakama.web-corpora.net}}: коми-зырянский, луговой марийский, мокшанский, удмуртский, эрзянский~\cite{Arkhangelskiy2020}. Удмуртский корпус разрабатывался в соавторстве с М. Медведевой, данные для звуковой части удмуртского корпуса собрала Е. Георгиева. Корпусы также включают тексты, извлечённые из публичных записей в соцсетях (в основном ВКонтакте), поэтому представленный язык близок к разговорной речи. Для каждого языка был разработан морфологический анализатор на основе правил. Анализаторы работают по данным словарей, поэтому в корпусах не распознаны несловарные слова или слова с опечатками. Проанализировано от 80\% до 96\% слов в корпусах. В основном контекст при морфологическом анализе не учитывается и анализатор выдаёт все возможные леммы для данной словоформы~\cite[58--59]{Arkhangelskiy2020}. Для малоресурсных языков важно, чтобы при автоматическом морфологическом анализе сохранялись в разметке все возможные формы слова для последующей проверки и выбора правильной формы лингвистом~\cite[61]{Arkhangelskiy2020}. Архангельский Т. называет такой корпусный менеджер в корпусе «дружественным к неоднозначности».

Жанровое разнообразие этих пяти корпусов таково. Основную массу составляют тексты электронных газет и журналов. Также корпусы включают художественную литературу, научные статьи, переводы Библии, статьи Википедии и официальные тексты. Большинство текстов написаны в 2010--2019 годах~\cite[59]{Arkhangelskiy2020}.

\subsubsection{Томский диалектный корпус}
Очень детальная семантическая разметка по жанрам и тематикам проведена в Томском диалектном корпусе~\cite{Zemicheva2019}.
Томский  диалектный  корпус\footnote{ См. http://losl.tsu.ru/corpus/demo}  создаётся  с  2017  г.  на  материале  диалектологических  экспедиций  в  среднеобский  регион  (Томская,  центральная  часть  Кемеровской  области).  Объём  ресурса на  апрель 2020 года более 1.7 млн словоупотреблений. Корпус имеет большой временной охват (70 лет) и детальную тематическая разметка (73  темы). 
 
%
% Компьютерная морфология
\section{Компьютерная морфология}\label{sect_review_comp_morphology}

Морфология -- это раздел лингвистики, изучающий структуру слова и его грамматические значения~\cite{MitreninaNikolaevLando2016}. Другими словами, морфология изучает
1) часть речи,
2) словоизменение,
3) словообразование,
4) грамматическое значение (что слово означает в предложении). 

Компьютерная морфология анализирует и синтезирует слова программными средствами~\cite{MitreninaNikolaevLando2016}. 

Грамматическое значение представляется в виде набора граммем. 
Граммемы группируются по категориям (падеж, время и т.д.). 
Одна и та же форма слова не может иметь две граммемы одной категории. 
С другой стороны формы могут совпадать, и их нужно уметь различать. 
Это одна из задач морфологии.

\TODO{Todo: Добавить сюда адаптацию для нашего языка рисунка из статьи Гарри ``Fine-grained Morphosyntactic Analysis and Generation Tools for More Than One Thousand Languages''}

Под \textbf{морфологическим анализом} (morphological analysis) 
подразумевается определение леммы (базовой, канонической нормы слова) и ее грамматических характеристик~\cite{MitreninaNikolaevLando2016}.

\nata{Есть определение для inflectional morphology?}
Отметим, что в ряде языков отсутствует морфологическое словоизменение (inflectional morphology), 
например, в языках йоруба и севернокитайском \TODO{(Vylomova et al., 2020, Todo ref: SIGMORPHON2020, page 2)}.


\subsection{Вепсский и карельский языки}\label{sect_review_veps_karelian}

Вепсский и карельский языки являются языками с богатой морфологией. \TODO{TODO: привести число словоформ в парадигме именной и глагольной формы языков. В виде таблицы? Ссылка на нашу статью?}

\cite{silfverberg2016finnpos}.




\subsection{Обзор компьютерных программ для морфологической обработки}

\subsection{Финский язык}\label{sect_review_fin}

Статья о лемматизаторе FinnPos~\cite{silfverberg2016finnpos}.

%https://github.com/mpsilfve/FinnPos

%
% Конечный преобразователь (трансдьюсер) и конечные автоматы
\section{Конечные преобразователи и автоматы} \label{sect_review_automaton}

Математические основы конечных автоматов и трансдьюсеров были разработаны 
несколько десятилетий назад~\cite{MohriChapter4Lothaire2005applied}.
%Lothaire2005applied}.


Вилфред Брауэр. Введение в теорию конечных автоматов. М.: Радио и Связь 392 с; 
1987 г. \todo{Прочитать (см. отечеств. терминологию, в библиотеке или libex).}

%Разработано программное обеспечение wcorpus~\cite{vakbib_soft_wcorpus}.


\subsection{Взвешенные трансдьюсеры} \label{sect_weighted_transducers}

Трансдьюсеры могут использоваться для отображения и связывания разных видов данных, 
например, слова и последовательности фонем. 
Подобные автоматы нужны при разработке систем распознавания речи~\cite[с.~200]{MohriChapter4Lothaire2005applied}.

Веса позволяют указать в такой модели отображения наличие неопределённости. 
Например, во взвешенных трансдьюсерах одному слову могут отвечать несколько вариантов 
произношения слова с разными рангами или вероятностями~\cite[с.~200]{MohriChapter4Lothaire2005applied}.


Базовые положения о морфологических трансдьюсерах в статье "Carlson, L., 2005. Inducing a morphological transducer from inflectional paradigms. Inquiries into Words, Constraints and Contexts, p.18-24."
%см. /data/all/docs/science/linguistics/finite-State-Transducers/about/morpho_from_paradigm_2005_Lauri_Carlson_42-48.pdf


История вопроса о конечных преобразователях с 1980 по 2005 год.
\todo{Karttunen, L. and Beesley, K.R., 2005. Twenty-five years of finite-state morphology. Inquiries Into Words, a Festschrift for Kimmo Koskenniemi on his 60th Birthday, pp.71-83.}
% 25years_FSMorphology_KarttunenBeesley2005.pdf

И в целом история вычислительной лингвистики \todo{Karttunen, L., 2007. Word play. Computational Linguistics, 33(4), pp.443-467.}
% Word_play_2007Karttunen.pdf

% As many observers have indicated, the most promising approaches will probably integrate rule-based and corpus-based methods. 
Как отмечают многие наблюдатели, наиболее многообещающие подходы, вероятно, будут включать методы, основанные на правилах и корпусе~\cite{Hutchins1999}.



%\section{Software automaton} \label{sect_automaton_soft}

%В статье сотрудников Google~\cite{Prasad2018}  перечислены открытые ресурсы для языков мира. 

%и \emph{Analogical grids}. 




%
% Нейронные сети
\section{Нейронные сети} \label{sect_nn}

\subsection{Введение} \label{sect_nn_review}

Глубинные нейронные сети -- это рычаг, который, опираясь на большие массивы данных, сдвинул многие камешки-задачи в лингвистике~\cite[2827]{Cruz-Anastasopoulos-Stump2020Chatino}.

 
%
% Постановка задачи
\chapter{Постановка задачи} \label{chapt_problem_formulation}

\section{Объект исследования} \label{sect_problem_obj}

Объект исследования~--- это зависимость между леммой, словоформами 
и набором грамматических признаков (грамсет).
Задачи морфологического синтеза и анализа решаются для слов на карельском языке.
Таким образом, объектом математического моделирования 
является естественный язык. 


\section{Предмет исследования} \label{sect_predmet_obj}

Предмет исследования~--- это морфологический синтез и морфологический анализ, 
в том числе задача лемматизации. 

Предмет исследования непосредственно соотнесен 
с целями исследования~\cite{Martishina2001Object}, 
с тем чтобы построить морфологический парсер и лемматизатор карельского языка. 

\section{Формализация задачи морфологического анализа} \label{sect_formal}

Придумать символы, отношения, правила, чтобы можно было формально показать работу 
разработанных правил словоизменения.


%
% Глава с моделями
\chapter{Модели} \label{chapt_models}

\section{Введение} \label{sect_model_intro}
Академик Ю.~Д.~Апресян еще в 1981 году писал о необходимости проведения широких теоретических поисковых исследований в области формальных моделей языка~\cite{Apresjan1981}.

В последние годы под языковыми моделями понимают...~\cite{Petroni2019}. 

В диссертационной работе используются модели другого рода.



\section{Система правил словоизменения} \label{sect_rules_flextion}

Глава 1.(с. 20-25). Морфология на основе слов. Есть набор словоформ. 
Перебираем все варианты перехода 
от одной формы к другой, n*(n-1)/2 правил~\cite[20--25]{Albright2002stem}.

Чтобы ограничить чрезмерное количество правил, предлагаем и применяем эвристики:

Эвристика 1: пусть существет только одна базовая словоформа 
(остальные получаем с помощью правил из базовой). 
Тогда для n словоформ в парадигме будет n правил~\cite{Albright2002stem}.

Эвристика 2: пусть существет две базовых словоформы 
(остальные получаем с помощью правил из двух базовых). 
Тогда для n словоформ в парадигме будет ??? правил.

Прекрасная визуализация морфологических правил в виде графа 
(рис. на с. 21-22)~\cite[21--22]{Albright2002stem}: 
вершина~--- это словоформа, ребро~--- это правило.

Построим такую визуализацию для морфологических правил карельского языка. 




\section{Частотный анализ} \label{sect_freq}


Вот курсовик с анализом частоты букв (слогов?) для эрзянского языка.

На странице 6 (последнее предложение)
\url{https://nauchkor.ru/pubs/chastotnyy-analiz-finno-ugorskih-yazykov-rossii-5697aa605f1be742640000df#page=6}
мысль о том, чтобы подсчитать частотность слов в корпусе и указывать эту информацию в словаре.

Там же, страница 8. Подсчитать в корпусе ВепКар частоту символов для каждого из языков, 
подсчитать для русского в wcorpus. Сравнить.

% Вывод
\section{Выводы по главе \ref{chapt_models}} \label{sect_conclusion_models}

 

 
%
% Архитектура Dictorpus
\chapter{Архитектура} \label{chapt_arch}

\section{Архитектура программного комплекса Dictorpus} \label{sect_arch_soft}

\section{Архитектура базы данных Dictorpus} \label{sect_arch_db}

 
\chapter{Эксперименты} \label{chapt2}

\section{Построение обратного словаря} \label{sect2_1}

           % Глава 2
\chapter{Анализ результатов} \label{chapt3}

\section{Анализ обратного словаря} \label{sect3_1}

           % Глава 3
\section{Выводы по главе \ref{chapt_models}} \label{sect_conclusion_models}

      % Заключение
\include{Dissertation/acronyms}        % Список сокращений и условных обозначений
\chapter*{Словарь терминов}             % Заголовок
\addcontentsline{toc}{chapter}{Словарь терминов}  % Добавляем его в оглавление

\textbf{Говор} "--- функционирующая языковая система, которая может отличаться от систем других говоров своеобразием фонетических, грамматических, словообразовательных и лексических черт.~
\cite{NovakPenttonenRuuskanenSiilin2019}

\textbf{Диалект} "--- разновидность данного языка, употребляемая в качестве средства общения лицами, связанными тесной территориальной общностью.~
\cite{NovakPenttonenRuuskanenSiilin2019}

\textbf{Компьютерная модель} "--- определение и построение взаимосвязи между входными данными (текстом) и результатом (лемма, морфологические свойства).

\textbf{Лемма} "--- базовая, каноническая форма слова.

\textbf{Морфология} "--- раздел лингвистики, который изучает структуру слова и его грамматические значения~\cite{MitreninaNikolaevLando2016}.

\textbf{Морфологический анализ} "--- определение леммы и ее грамматических характеристик~\cite{MitreninaNikolaevLando2016}.

\textbf{Наречие} "--- наиболее крупная единица диалектного членения языка, совокупность диалектов, объединенных общими признаками.~
\cite{NovakPenttonenRuuskanenSiilin2019}

\textbf{Нормализация} "--- постановка слова или словосочетания в каноническую форму~\cite{MitreninaNikolaevLando2016}.

\textbf{Словоформа} "--- слово в определенной грамматической форме~\cite{MitreninaNikolaevLando2016}.

\textbf{Токенизация} "--- разбитие текста на предложения, а в каждом предложении выделение слов, знаков препинания и других элементов текста - числа, формулы, таблицы и т.д.~\cite{MitreninaNikolaevLando2016}.

\textbf{Токены} "--- выделенные в результате токенизации единицы (слова, числа, знаки препинания и пр.)~\cite{MitreninaNikolaevLando2016}.      % Словарь терминов
\include{Dissertation/references}      % Список литературы
\include{Dissertation/lists}           % Списки таблиц и изображений (иллюстративный материал)

\setcounter{totalchapter}{\value{chapter}} % Подсчёт количества глав

%%% Настройки для приложений
\appendix
% Оформление заголовков приложений ближе к ГОСТ:
\setlength{\midchapskip}{20pt}
\renewcommand*{\afterchapternum}{\par\nobreak\vskip \midchapskip}
\renewcommand\thechapter{\Asbuk{chapter}} % Чтобы приложения русскими буквами нумеровались

\chapter{Регистрация программ для ЭВМ} \label{AppendixA}

Свидетельства о государственной регистрации программы для ЭВМ.
%\graphicspath{{../images/}{images/}} 
\begin{figure}[!h]
    \centering
    \includegraphics[height=0.81\textheight,keepaspectratio=true]{svid_registr_wcorpus_2018_Krizhanovsky.jpg}
%   \caption{A boat.}
    \label{fig:svid_rospatent_wcorpus_2018}
\end{figure}

\begin{figure}[!h]
    \centering
    \includegraphics[height=0.81\textheight,keepaspectratio=true]{svid_wordform_generator_krl-proper_2019_Krizhanovsky_Novak.jpg}
%   \caption{A boat.}
    \label{fig:svid_rospatent_2019}
\end{figure}

\chapter{Очень длинное название второго приложения, в~котором продемонстрирована работа с~длинными таблицами} \label{AppendixB}

\section{Подраздел приложения}\label{AppendixB1}
Вот размещается длинная таблица:
\fontsize{10pt}{10pt}\selectfont
\begin{longtable*}[c]{|l|c|l|l|} %longtable* появляется из пакета ltcaption и даёт ненумерованную таблицу
% \caption{Описание входных файлов модели}\label{Namelists} 
%\\
 \hline
 %\multicolumn{4}{|c|}{\textbf{Файл puma\_namelist}}        \\ \hline
 Параметр & Умолч. & Тип & Описание               \\ \hline
                                              \endfirsthead   \hline
 \multicolumn{4}{|c|}{\small\slshape (продолжение)}        \\ \hline
 Параметр & Умолч. & Тип & Описание               \\ \hline
                                              \endhead        \hline
% \multicolumn{4}{|c|}{\small\slshape (окончание)}        \\ \hline
% Параметр & Умолч. & Тип & Описание               \\ \hline
%                                             \endlasthead        \hline
 \multicolumn{4}{|r|}{\small\slshape продолжение следует}  \\ \hline
                                              \endfoot        \hline
                                              \endlastfoot
 \multicolumn{4}{|l|}{\&INP}        \\ \hline 
 kick & 1 & int & 0: инициализация без шума ($p_s = const$) \\
      &   &     & 1: генерация белого шума                  \\
      &   &     & 2: генерация белого шума симметрично относительно \\
  & & & экватора    \\
 mars & 0 & int & 1: инициализация модели для планеты Марс     \\
 kick & 1 & int & 0: инициализация без шума ($p_s = const$) \\
      &   &     & 1: генерация белого шума                  \\
      &   &     & 2: генерация белого шума симметрично относительно \\
  & & & экватора    \\
 mars & 0 & int & 1: инициализация модели для планеты Марс     \\
kick & 1 & int & 0: инициализация без шума ($p_s = const$) \\
      &   &     & 1: генерация белого шума                  \\
      &   &     & 2: генерация белого шума симметрично относительно \\
  & & & экватора    \\
 mars & 0 & int & 1: инициализация модели для планеты Марс     \\
kick & 1 & int & 0: инициализация без шума ($p_s = const$) \\
      &   &     & 1: генерация белого шума                  \\
      &   &     & 2: генерация белого шума симметрично относительно \\
  & & & экватора    \\
 mars & 0 & int & 1: инициализация модели для планеты Марс     \\
kick & 1 & int & 0: инициализация без шума ($p_s = const$) \\
      &   &     & 1: генерация белого шума                  \\
      &   &     & 2: генерация белого шума симметрично относительно \\
  & & & экватора    \\
 mars & 0 & int & 1: инициализация модели для планеты Марс     \\
kick & 1 & int & 0: инициализация без шума ($p_s = const$) \\
      &   &     & 1: генерация белого шума                  \\
      &   &     & 2: генерация белого шума симметрично относительно \\
  & & & экватора    \\
 mars & 0 & int & 1: инициализация модели для планеты Марс     \\
kick & 1 & int & 0: инициализация без шума ($p_s = const$) \\
      &   &     & 1: генерация белого шума                  \\
      &   &     & 2: генерация белого шума симметрично относительно \\
  & & & экватора    \\
 mars & 0 & int & 1: инициализация модели для планеты Марс     \\
kick & 1 & int & 0: инициализация без шума ($p_s = const$) \\
      &   &     & 1: генерация белого шума                  \\
      &   &     & 2: генерация белого шума симметрично относительно \\
  & & & экватора    \\
 mars & 0 & int & 1: инициализация модели для планеты Марс     \\
kick & 1 & int & 0: инициализация без шума ($p_s = const$) \\
      &   &     & 1: генерация белого шума                  \\
      &   &     & 2: генерация белого шума симметрично относительно \\
  & & & экватора    \\
 mars & 0 & int & 1: инициализация модели для планеты Марс     \\
kick & 1 & int & 0: инициализация без шума ($p_s = const$) \\
      &   &     & 1: генерация белого шума                  \\
      &   &     & 2: генерация белого шума симметрично относительно \\
  & & & экватора    \\
 mars & 0 & int & 1: инициализация модели для планеты Марс     \\
kick & 1 & int & 0: инициализация без шума ($p_s = const$) \\
      &   &     & 1: генерация белого шума                  \\
      &   &     & 2: генерация белого шума симметрично относительно \\
  & & & экватора    \\
 mars & 0 & int & 1: инициализация модели для планеты Марс     \\
kick & 1 & int & 0: инициализация без шума ($p_s = const$) \\
      &   &     & 1: генерация белого шума                  \\
      &   &     & 2: генерация белого шума симметрично относительно \\
  & & & экватора    \\
 mars & 0 & int & 1: инициализация модели для планеты Марс     \\
kick & 1 & int & 0: инициализация без шума ($p_s = const$) \\
      &   &     & 1: генерация белого шума                  \\
      &   &     & 2: генерация белого шума симметрично относительно \\
  & & & экватора    \\
 mars & 0 & int & 1: инициализация модели для планеты Марс     \\
kick & 1 & int & 0: инициализация без шума ($p_s = const$) \\
      &   &     & 1: генерация белого шума                  \\
      &   &     & 2: генерация белого шума симметрично относительно \\
  & & & экватора    \\
 mars & 0 & int & 1: инициализация модели для планеты Марс     \\
kick & 1 & int & 0: инициализация без шума ($p_s = const$) \\
      &   &     & 1: генерация белого шума                  \\
      &   &     & 2: генерация белого шума симметрично относительно \\
  & & & экватора    \\
 mars & 0 & int & 1: инициализация модели для планеты Марс     \\
 \hline
  %& & & $\:$ \\ 
 \multicolumn{4}{|l|}{\&SURFPAR}        \\ \hline
kick & 1 & int & 0: инициализация без шума ($p_s = const$) \\
      &   &     & 1: генерация белого шума                  \\
      &   &     & 2: генерация белого шума симметрично относительно \\
  & & & экватора    \\
 mars & 0 & int & 1: инициализация модели для планеты Марс     \\
kick & 1 & int & 0: инициализация без шума ($p_s = const$) \\
      &   &     & 1: генерация белого шума                  \\
      &   &     & 2: генерация белого шума симметрично относительно \\
  & & & экватора    \\
 mars & 0 & int & 1: инициализация модели для планеты Марс     \\
kick & 1 & int & 0: инициализация без шума ($p_s = const$) \\
      &   &     & 1: генерация белого шума                  \\
      &   &     & 2: генерация белого шума симметрично относительно \\
  & & & экватора    \\
 mars & 0 & int & 1: инициализация модели для планеты Марс     \\
kick & 1 & int & 0: инициализация без шума ($p_s = const$) \\
      &   &     & 1: генерация белого шума                  \\
      &   &     & 2: генерация белого шума симметрично относительно \\
  & & & экватора    \\
 mars & 0 & int & 1: инициализация модели для планеты Марс     \\
kick & 1 & int & 0: инициализация без шума ($p_s = const$) \\
      &   &     & 1: генерация белого шума                  \\
      &   &     & 2: генерация белого шума симметрично относительно \\
  & & & экватора    \\
 mars & 0 & int & 1: инициализация модели для планеты Марс     \\
kick & 1 & int & 0: инициализация без шума ($p_s = const$) \\
      &   &     & 1: генерация белого шума                  \\
      &   &     & 2: генерация белого шума симметрично относительно \\
  & & & экватора    \\
 mars & 0 & int & 1: инициализация модели для планеты Марс     \\
kick & 1 & int & 0: инициализация без шума ($p_s = const$) \\
      &   &     & 1: генерация белого шума                  \\
      &   &     & 2: генерация белого шума симметрично относительно \\
  & & & экватора    \\
 mars & 0 & int & 1: инициализация модели для планеты Марс     \\
kick & 1 & int & 0: инициализация без шума ($p_s = const$) \\
      &   &     & 1: генерация белого шума                  \\
      &   &     & 2: генерация белого шума симметрично относительно \\
  & & & экватора    \\
 mars & 0 & int & 1: инициализация модели для планеты Марс     \\
kick & 1 & int & 0: инициализация без шума ($p_s = const$) \\
      &   &     & 1: генерация белого шума                  \\
      &   &     & 2: генерация белого шума симметрично относительно \\
  & & & экватора    \\
 mars & 0 & int & 1: инициализация модели для планеты Марс     \\ 
 \hline 
\end{longtable*}

\normalsize% возвращаем шрифт к нормальному
\section{Ещё один подраздел приложения} \label{AppendixB2}

Нужно больше подразделов приложения!
Конвынёры витюпырата но нам, тебиквюэ мэнтётюм позтюлант ед про. Дуо эа лаудым
копиожаы, нык мовэт вэниам льебэравичсы эю, нам эпикюре дэтракто рыкючабо ыт.

Пример длинной таблицы с записью продолжения по ГОСТ 2.105:

\begingroup
    \centering
    \small
    \begin{longtable}[c]{|l|c|l|l|}
    \caption{Наименование таблицы средней длины}%
    \label{tbl:test5}% label всегда желательно идти после caption
    \\[-0.45\onelineskip]
    \hline
    Параметр & Умолч. & Тип & Описание\\ \hline
    \endfirsthead%
    \caption*{\tabcapalign Продолжение таблицы~\thetable}\\[-0.45\onelineskip]
    \hline
    Параметр & Умолч. & Тип & Описание\\ \hline
    \endhead
    \hline
    \endfoot
    \hline
     \endlastfoot
     \multicolumn{4}{|l|}{\&INP}        \\ \hline 
     kick & 1 & int & 0: инициализация без шума ($p_s = const$) \\
          &   &     & 1: генерация белого шума                  \\
          &   &     & 2: генерация белого шума симметрично относительно \\
      & & & экватора    \\
     mars & 0 & int & 1: инициализация модели для планеты Марс     \\
     kick & 1 & int & 0: инициализация без шума ($p_s = const$) \\
          &   &     & 1: генерация белого шума                  \\
          &   &     & 2: генерация белого шума симметрично относительно \\
      & & & экватора    \\
     mars & 0 & int & 1: инициализация модели для планеты Марс     \\
    kick & 1 & int & 0: инициализация без шума ($p_s = const$) \\
          &   &     & 1: генерация белого шума                  \\
          &   &     & 2: генерация белого шума симметрично относительно \\
      & & & экватора    \\
     mars & 0 & int & 1: инициализация модели для планеты Марс     \\
    kick & 1 & int & 0: инициализация без шума ($p_s = const$) \\
          &   &     & 1: генерация белого шума                  \\
          &   &     & 2: генерация белого шума симметрично относительно \\
      & & & экватора    \\
     mars & 0 & int & 1: инициализация модели для планеты Марс     \\
    kick & 1 & int & 0: инициализация без шума ($p_s = const$) \\
          &   &     & 1: генерация белого шума                  \\
          &   &     & 2: генерация белого шума симметрично относительно \\
      & & & экватора    \\
     mars & 0 & int & 1: инициализация модели для планеты Марс     \\
    kick & 1 & int & 0: инициализация без шума ($p_s = const$) \\
          &   &     & 1: генерация белого шума                  \\
          &   &     & 2: генерация белого шума симметрично относительно \\
      & & & экватора    \\
     mars & 0 & int & 1: инициализация модели для планеты Марс     \\
    kick & 1 & int & 0: инициализация без шума ($p_s = const$) \\
          &   &     & 1: генерация белого шума                  \\
          &   &     & 2: генерация белого шума симметрично относительно \\
      & & & экватора    \\
     mars & 0 & int & 1: инициализация модели для планеты Марс     \\
    kick & 1 & int & 0: инициализация без шума ($p_s = const$) \\
          &   &     & 1: генерация белого шума                  \\
          &   &     & 2: генерация белого шума симметрично относительно \\
      & & & экватора    \\
     mars & 0 & int & 1: инициализация модели для планеты Марс     \\
    kick & 1 & int & 0: инициализация без шума ($p_s = const$) \\
          &   &     & 1: генерация белого шума                  \\
          &   &     & 2: генерация белого шума симметрично относительно \\
      & & & экватора    \\
     mars & 0 & int & 1: инициализация модели для планеты Марс     \\
    kick & 1 & int & 0: инициализация без шума ($p_s = const$) \\
          &   &     & 1: генерация белого шума                  \\
          &   &     & 2: генерация белого шума симметрично относительно \\
      & & & экватора    \\
     mars & 0 & int & 1: инициализация модели для планеты Марс     \\
    kick & 1 & int & 0: инициализация без шума ($p_s = const$) \\
          &   &     & 1: генерация белого шума                  \\
          &   &     & 2: генерация белого шума симметрично относительно \\
      & & & экватора    \\
     mars & 0 & int & 1: инициализация модели для планеты Марс     \\
    kick & 1 & int & 0: инициализация без шума ($p_s = const$) \\
          &   &     & 1: генерация белого шума                  \\
          &   &     & 2: генерация белого шума симметрично относительно \\
      & & & экватора    \\
     mars & 0 & int & 1: инициализация модели для планеты Марс     \\
    kick & 1 & int & 0: инициализация без шума ($p_s = const$) \\
          &   &     & 1: генерация белого шума                  \\
          &   &     & 2: генерация белого шума симметрично относительно \\
      & & & экватора    \\
     mars & 0 & int & 1: инициализация модели для планеты Марс     \\
    kick & 1 & int & 0: инициализация без шума ($p_s = const$) \\
          &   &     & 1: генерация белого шума                  \\
          &   &     & 2: генерация белого шума симметрично относительно \\
      & & & экватора    \\
     mars & 0 & int & 1: инициализация модели для планеты Марс     \\
    kick & 1 & int & 0: инициализация без шума ($p_s = const$) \\
          &   &     & 1: генерация белого шума                  \\
          &   &     & 2: генерация белого шума симметрично относительно \\
      & & & экватора    \\
     mars & 0 & int & 1: инициализация модели для планеты Марс     \\
     \hline
      %& & & $\:$ \\ 
     \multicolumn{4}{|l|}{\&SURFPAR}        \\ \hline
    kick & 1 & int & 0: инициализация без шума ($p_s = const$) \\
          &   &     & 1: генерация белого шума                  \\
          &   &     & 2: генерация белого шума симметрично относительно \\
      & & & экватора    \\
     mars & 0 & int & 1: инициализация модели для планеты Марс     \\
    kick & 1 & int & 0: инициализация без шума ($p_s = const$) \\
          &   &     & 1: генерация белого шума                  \\
          &   &     & 2: генерация белого шума симметрично относительно \\
      & & & экватора    \\
     mars & 0 & int & 1: инициализация модели для планеты Марс     \\
    kick & 1 & int & 0: инициализация без шума ($p_s = const$) \\
          &   &     & 1: генерация белого шума                  \\
          &   &     & 2: генерация белого шума симметрично относительно \\
      & & & экватора    \\
     mars & 0 & int & 1: инициализация модели для планеты Марс     \\
    kick & 1 & int & 0: инициализация без шума ($p_s = const$) \\
          &   &     & 1: генерация белого шума                  \\
          &   &     & 2: генерация белого шума симметрично относительно \\
      & & & экватора    \\
     mars & 0 & int & 1: инициализация модели для планеты Марс     \\
    kick & 1 & int & 0: инициализация без шума ($p_s = const$) \\
          &   &     & 1: генерация белого шума                  \\
          &   &     & 2: генерация белого шума симметрично относительно \\
      & & & экватора    \\
     mars & 0 & int & 1: инициализация модели для планеты Марс     \\
    kick & 1 & int & 0: инициализация без шума ($p_s = const$) \\
          &   &     & 1: генерация белого шума                  \\
          &   &     & 2: генерация белого шума симметрично относительно \\
      & & & экватора    \\
     mars & 0 & int & 1: инициализация модели для планеты Марс     \\
    kick & 1 & int & 0: инициализация без шума ($p_s = const$) \\
          &   &     & 1: генерация белого шума                  \\
          &   &     & 2: генерация белого шума симметрично относительно \\
      & & & экватора    \\
     mars & 0 & int & 1: инициализация модели для планеты Марс     \\
    kick & 1 & int & 0: инициализация без шума ($p_s = const$) \\
          &   &     & 1: генерация белого шума                  \\
          &   &     & 2: генерация белого шума симметрично относительно \\
      & & & экватора    \\
     mars & 0 & int & 1: инициализация модели для планеты Марс     \\
    kick & 1 & int & 0: инициализация без шума ($p_s = const$) \\
          &   &     & 1: генерация белого шума                  \\
          &   &     & 2: генерация белого шума симметрично относительно \\
      & & & экватора    \\
     mars & 0 & int & 1: инициализация модели для планеты Марс     \\ 
    \end{longtable}
\normalsize% возвращаем шрифт к нормальному
\endgroup



\section{Форматирование внутри таблиц} \label{AppendixB3}

В таблице~\ref{tbl:other-row} пример с чересстрочным
форматированием. В~файле \verb+userstyles.tex+  задаётся счётчик
\verb+\newcounter{rowcnt}+ который увеличивается на~1 после каждой
строчки (как указано в преамбуле таблицы). Кроме того, задаётся
условный макрос \verb+\altshape+ который выдаёт одно
из~двух типов форматирования в~зависимости от чётности счётчика.

В таблице~\ref{tbl:other-row} каждая чётная строчка "--- синяя,
нечётная "--- с наклоном и~слегка поднята вверх. Визуально это приводит
к тому, что среднее значение и~среднеквадратичное изменение
группируются и хорошо выделяются взглядом в~таблице. Сохраняется
возможность отдельные значения в таблице выделить цветом или
шрифтом. К первому и второму столбцу форматирование не применяется
по~сути таблицы, к шестому общее форматирование не~применяется для
наглядности.

Так как заголовок таблицы тоже считается за строчку, то перед ним (для
первого, промежуточного и финального варианта) счётчик обнуляется,
а~в~\verb+\altshape+ для нулевого значения счётчика форматирования
не~применяется.

\begingroup % Ограничиваем область видимости arraystretch
\renewcommand\altshape{
  \ifnumequal{\value{rowcnt}}{0}{
    % Стиль для заголовка таблицы
  }{
    \ifnumodd{\value{rowcnt}}
    {
      \color{blue} % Cтиль для нечётных строк
    }{
      \vspace*{-0.7ex}\itshape} % Стиль для чётных строк
  }
}
\newcolumntype{A}{ >{\altshape}X[1mc]}
\needspace{2\baselineskip}
\renewcommand{\arraystretch}{0.9}%% Уменьшаем  расстояние между
                                %% рядами, чтобы таблица не так много
                                %% места занимала в дисере.
\begin{longtabu} to \textwidth {@{}X[0.1ml]X[0.3mc]A@{}A@{}A@{}X[0.98mc]@{}>{\setlength{\baselineskip}{0.7\baselineskip}}A@{}A<{\stepcounter{rowcnt}}@{}}
% \begin{longtabu} to \textwidth {@{}X[0.2ml]X[1mc]X[1mc]X[1mc]X[1mc]X[1mc]>{\setlength{\baselineskip}{0.7\baselineskip}}X[1mc]X[1mc]@{}}
  \caption{Длинная таблица с примером чересстрочного форматирования\label{tbl:other-row}}\vspace*{1ex}\\% label всегда желательно идти после caption
  % \vspace*{1ex}     \\

  \toprule %%% верхняя линейка  
\setcounter{rowcnt}{0} &Итера\-ции & JADE\texttt{++} & JADE & jDE & SaDE
& DE/rand /1/bin & PSO \\ 
 \midrule %%% тонкий разделитель. Отделяет названия столбцов. Обязателен по ГОСТ 2.105 пункт 4.4.5 
 \endfirsthead

 \multicolumn{8}{c}{\small\slshape (продолжение)} \\ 
 \toprule %%% верхняя линейка
\setcounter{rowcnt}{0} &Итера\-ции & JADE\texttt{++} & JADE & jDE & SaDE
& DE/rand /1/bin & PSO \\ 
 \midrule %%% тонкий разделитель. Отделяет названия столбцов. Обязателен по ГОСТ 2.105 пункт 4.4.5 
 \endhead
 
 \multicolumn{8}{c}{\small\slshape (окончание)} \\ 
 \toprule %%% верхняя линейка
\setcounter{rowcnt}{0} &Итера\-ции & JADE\texttt{++} & JADE & jDE & SaDE
& DE/rand /1/bin & PSO \\ 
 \midrule %%% тонкий разделитель. Отделяет названия столбцов. Обязателен по ГОСТ 2.105 пункт 4.4.5 
 \endlasthead

 \bottomrule %%% нижняя линейка
 \multicolumn{8}{r}{\small\slshape продолжение следует}     \\ 
 \endfoot 
 \endlastfoot
 
f1  & 1500 & \textbf{1.8E-60}   & 1.3E-54   & 2.5E-28   & 4.5E-20   & 9.8E-14   & 9.6E-42   \\\nopagebreak
    &      & (8.4E-60) & (9.2E-54) & \color{red}(3.5E-28) & (6.9E-20) & (8.4E-14) & (2.7E-41) \\
f2  & 2000 & 1.8E-25   & 3.9E-22   & 1.5E-23   & 1.9E-14   & 1.6E-09   & 9.3E-21   \\\nopagebreak
    &      & (8.8E-25) & (2.7E-21) & (1.0E-23) & (1.1E-14) & (1.1E-09) & (6.3E-20) \\
f3  & 5000 & 5.7E-61   & 6.0E-87   & 5.2E-14   & \color{green}9.0E-37   & 6.6E-11   & 2.5E-19   \\\nopagebreak
    &      & (2.7E-60) & (1.9E-86) & (1.1E-13) & (5.4E-36) & (8.8E-11) & (3.9E-19) \\
f4  & 5000 & 8.2E-24   & 4.3E-66   & 1.4E-15   & 7.4E-11   & 4.2E-01   & 4.4E-14   \\\nopagebreak
    &      & (4.0E-23) & (1.2E-65) & (1.0E-15) & (1.8E-10) & (1.1E+00) & (9.3E-14) \\
f5  & 3000 & 8.0E-02   & 3.2E-01   & 1.3E+01   & 2.1E+01   & 2.1E+00   & 2.5E+01   \\\nopagebreak
    &      & (5.6E-01) & (1.1E+00) & (1.4E+01) & (7.8E+00) & (1.5E+00) & (3.2E+01) \\
f6  & 100  & 2.9E+00   & 5.6E+00   & 1.0E+03   & 9.3E+02   & 4.7E+03   & 4.5E+01   \\\nopagebreak
    &      & (1.2E+00) & (1.6E+00) & (2.2E+02) & (1.8E+02) & (1.1E+03) & (2.4E+01) \\
f7  & 3000 & 6.4E-04   & 6.8E-04   & 3.3E-03   & 4.8E-03   & 4.7E-03   & 2.5E-03   \\\nopagebreak
    &      & (2.5E-04) & (2.5E-04) & (8.5E-04) & (1.2E-03) & (1.2E-03) & (1.4E-03) \\
f8  & 1000 & 3.3E-05   & 7.1E+00   & 7.9E-11   & 4.7E+00   & 5.9E+03   & 2.4E+03   \\\nopagebreak
    &      & (2.3E-05) & (2.8E+01) & (1.3E-10) & (3.3E+01) & (1.1E+03) & (6.7E+02) \\
f9  & 1000 & 1.0E-04   & 1.4E-04   & 1.5E-04   & 1.2E-03   & 1.8E+02   & 5.2E+01   \\\nopagebreak
    &      & (6.0E-05) & (6.5E-05) & (2.0E-04) & (6.5E-04) & (1.3E+01) & (1.6E+01) \\
f10 & 500  & 8.2E-10   & 3.0E-09   & 3.5E-04   & 2.7E-03   & 1.1E-01   & 4.6E-01   \\\nopagebreak
    &      & (6.9E-10) & (2.2E-09) & (1.0E-04) & (5.1E-04) & (3.9E-02) & (6.6E-01) \\
f11 & 500  & 9.9E-08   & 2.0E-04   & 1.9E-05   & 7.8E-04)  & 2.0E-01   & 1.3E-02   \\\nopagebreak
    &      & (6.0E-07) & (1.4E-03) & (5.8E-05) & (1.2E-03  & (1.1E-01) & (1.7E-02) \\
f12 & 500  & 4.6E-17   & 3.8E-16   & 1.6E-07   & 1.9E-05   & 1.2E-02   & 1.9E-01   \\\nopagebreak
    &      & (1.9E-16) & (8.3E-16) & (1.5E-07) & (9.2E-06) & (1.0E-02) & (3.9E-01) \\
f13 & 500  & 2.0E-16   & 1.2E-15   & 1.5E-06   & 6.1E-05   & 7.5E-02   & 2.9E-03   \\\nopagebreak
    &      & (6.5E-16) & (2.8E-15) & (9.8E-07) & (2.0E-05) & (3.8E-02) & (4.8E-03) \\
f1  & 1500 & \textbf{1.8E-60}   & 1.3E-54   & 2.5E-28   & 4.5E-20   & 9.8E-14   & 9.6E-42   \\\nopagebreak
    &      & (8.4E-60) & (9.2E-54) & \color{red}(3.5E-28) & (6.9E-20) & (8.4E-14) & (2.7E-41) \\
f2  & 2000 & 1.8E-25   & 3.9E-22   & 1.5E-23   & 1.9E-14   & 1.6E-09   & 9.3E-21   \\\nopagebreak
    &      & (8.8E-25) & (2.7E-21) & (1.0E-23) & (1.1E-14) & (1.1E-09) & (6.3E-20) \\
f3  & 5000 & 5.7E-61   & 6.0E-87   & 5.2E-14   & 9.0E-37   & 6.6E-11   & 2.5E-19   \\\nopagebreak
    &      & (2.7E-60) & (1.9E-86) & (1.1E-13) & (5.4E-36) & (8.8E-11) & (3.9E-19) \\
f4  & 5000 & 8.2E-24   & 4.3E-66   & 1.4E-15   & 7.4E-11   & 4.2E-01   & 4.4E-14   \\\nopagebreak
    &      & (4.0E-23) & (1.2E-65) & (1.0E-15) & (1.8E-10) & (1.1E+00) & (9.3E-14) \\
f5  & 3000 & 8.0E-02   & 3.2E-01   & 1.3E+01   & 2.1E+01   & 2.1E+00   & 2.5E+01   \\\nopagebreak
    &      & (5.6E-01) & (1.1E+00) & (1.4E+01) & (7.8E+00) & (1.5E+00) & (3.2E+01) \\
f6  & 100  & 2.9E+00   & 5.6E+00   & 1.0E+03   & 9.3E+02   & 4.7E+03   & 4.5E+01   \\\nopagebreak
    &      & (1.2E+00) & (1.6E+00) & (2.2E+02) & (1.8E+02) & (1.1E+03) & (2.4E+01) \\
f7  & 3000 & 6.4E-04   & 6.8E-04   & 3.3E-03   & 4.8E-03   & 4.7E-03   & 2.5E-03   \\\nopagebreak
    &      & (2.5E-04) & (2.5E-04) & (8.5E-04) & (1.2E-03) & (1.2E-03) & (1.4E-03) \\
f8  & 1000 & 3.3E-05   & 7.1E+00   & 7.9E-11   & 4.7E+00   & 5.9E+03   & 2.4E+03   \\\nopagebreak
    &      & (2.3E-05) & (2.8E+01) & (1.3E-10) & (3.3E+01) & (1.1E+03) & (6.7E+02) \\
f9  & 1000 & 1.0E-04   & 1.4E-04   & 1.5E-04   & 1.2E-03   & 1.8E+02   & 5.2E+01   \\\nopagebreak
    &      & (6.0E-05) & (6.5E-05) & (2.0E-04) & (6.5E-04) & (1.3E+01) & (1.6E+01) \\
f10 & 500  & 8.2E-10   & 3.0E-09   & 3.5E-04   & 2.7E-03   & 1.1E-01   & 4.6E-01   \\\nopagebreak
    &      & (6.9E-10) & (2.2E-09) & (1.0E-04) & (5.1E-04) & (3.9E-02) & (6.6E-01) \\
f11 & 500  & 9.9E-08   & 2.0E-04   & 1.9E-05   & 7.8E-04)  & 2.0E-01   & 1.3E-02   \\\nopagebreak
    &      & (6.0E-07) & (1.4E-03) & (5.8E-05) & (1.2E-03  & (1.1E-01) & (1.7E-02) \\
f12 & 500  & 4.6E-17   & 3.8E-16   & 1.6E-07   & 1.9E-05   & 1.2E-02   & 1.9E-01   \\\nopagebreak
    &      & (1.9E-16) & (8.3E-16) & (1.5E-07) & (9.2E-06) & (1.0E-02) & (3.9E-01) \\
f13 & 500  & 2.0E-16   & 1.2E-15   & 1.5E-06   & 6.1E-05   & 7.5E-02   & 2.9E-03   \\\nopagebreak
    &      & (6.5E-16) & (2.8E-15) & (9.8E-07) & (2.0E-05) & (3.8E-02) & (4.8E-03) \\
\bottomrule %%% нижняя линейка
\end{longtabu} \endgroup

\section{Очередной подраздел приложения} \label{AppendixB4}

Нужно больше подразделов приложения!

\section{И ещё один подраздел приложения} \label{AppendixB5}

Нужно больше подразделов приложения!
        % Приложения

\setcounter{totalappendix}{\value{chapter}} % Подсчёт количества приложений

\end{document}
