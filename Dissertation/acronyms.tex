\chapter*{Список сокращений и условных обозначений} % Заголовок
\addcontentsline{toc}{chapter}{Список сокращений и условных обозначений}  % Добавляем его в оглавление
\noindent
%\begin{longtabu} to \dimexpr \textwidth-5\tabcolsep {r X}
\begin{longtabu} to \textwidth {r X}
% Жирное начертание для математических символов может иметь
% дополнительный смысл, поэтому они приводятся как в тексте
% диссертации
\(\begin{rcases}
a_n\\
b_n
\end{rcases}\)  &
\begin{minipage}{\linewidth}
коэффициенты разложения Ми в дальнем поле соответствующие
электрическим и магнитным мультиполям
\end{minipage}
\\
\({\boldsymbol{\hat{\mathrm e}}}\) & единичный вектор \\
\(E_0\) & амплитуда падающего поля\\
\(\begin{rcases}
a_n\\
b_n
\end{rcases}\)  &
коэффициенты разложения Ми в дальнем поле соответствующие
электрическим и магнитным мультиполям ещё раз, но~без окружения
minipage нет вертикального выравнивания по~центру.
\\
\(j\) & тип функции Бесселя\\
\(k\) & волновой вектор падающей волны\\

\(\begin{rcases}
a_n\\
b_n
\end{rcases}\)  &
\begin{minipage}{\linewidth}
\vspace{0.7em}
и снова коэффициенты разложения Ми в дальнем поле соответствующие
электрическим и магнитным мультиполям, теперь окружение minipage есть
и добавлено много текста, так что описание группы условных
обозначений значительно превысило высоту этой группы... Для отбивки
пришлось добавить дополнительные отступы.
\vspace{0.5em}
\end{minipage}
\\
\(L\) & общее число слоёв\\
\(l\) & номер слоя внутри стратифицированной сферы\\
\(\lambda\) & длина волны электромагнитного излучения
в вакууме\\
\(n\) & порядок мультиполя\\
\(\begin{rcases}
{\mathbf{N}}_{e1n}^{(j)}&{\mathbf{N}}_{o1n}^{(j)}\\
{\mathbf{M}_{o1n}^{(j)}}&{\mathbf{M}_{e1n}^{(j)}}
\end{rcases}\)  & сферические векторные гармоники\\
\(\mu\)  & магнитная проницаемость в вакууме\\
\(r,\theta,\phi\) & полярные координаты\\
\(\omega\) & частота падающей волны\\

\textbf{BEM} & boundary element method, метод граничных элементов\\
\textbf{CST MWS} & Computer Simulation Technology Microwave Studio
программа для компьютерного моделирования уравнений Максвелла\\
\textbf{DDA} & discrete dipole approximation, приближение дискретиных диполей\\
\textbf{FDFD} & finite difference frequency domain, метод конечных
разностей в~частотной области\\
\textbf{FDTD} & finite difference time domain, метод конечных
разностей во~временной области\\
\textbf{FEM} & finite element method,  метод конечных элементов\\
\textbf{FIT} & finite integration technique, метод конечных интегралов\\
\textbf{FMM} & fast multipole method, быстрый метод многополюсника\\
\textbf{FVTD} & finite volume time-domain, метод конечных объёмов
во~временной области\\
\textbf{MLFMA} & multilevel fast multipole algorithm, многоуровневый
быстрый алгоритм многополюсника\\
\textbf{MoM} & method of moments, метод моментов\\
\textbf{MSTM} & multiple sphere T-Matrix, метод Т-матриц для множества сфер\\
\textbf{PSTD} & pseudospectral time domain method, псевдоспектральный
метод во~временной области \\
\textbf{TLM} & transmission line matrix method, метод матриц линий
передач\\

\end{longtabu}
\addtocounter{table}{-1}% Нужно откатить на единицу счетчик номеров таблиц, так как предыдующая таблица сделана для удобства представления информации по ГОСТ
