\section{Конечные преобразователи и автоматы} \label{sect_review_automaton}

Математические основы конечных автоматов и трансдьюсеров были разработаны 
несколько десятилетий назад~\cite{MohriChapter4Lothaire2005applied}.
%Lothaire2005applied}.


Вилфред Брауэр. Введение в теорию конечных автоматов. М.: Радио и Связь 392 с; 
1987 г. \todo{Прочитать (см. отечеств. терминологию, в библиотеке или libex).}

%Разработано программное обеспечение wcorpus~\cite{vakbib_soft_wcorpus}.



\subsection{Взвешенные трансдьюсеры} \label{sect_weighted_transducers}

Трансдьюсеры могут использоваться для отображения и связывания разных видов данных, 
например, слова и последовательности фонем. 
Подобные автоматы нужны при разработке систем распознавания речи~\cite[с.~200]{MohriChapter4Lothaire2005applied}.

Веса позволяют указать в такой модели отображения наличие неопределённости. 
Например, во взвешенных трансдьюсерах одному слову могут отвечать несколько вариантов 
произношения слова с разными рангами или вероятностями~\cite[с.~200]{MohriChapter4Lothaire2005applied}.


%\section{Software automaton} \label{sect_automaton_soft}

%В статье сотрудников Google~\cite{Prasad2018}  перечислены открытые ресурсы для языков мира. 

%и \emph{Analogical grids}. 



