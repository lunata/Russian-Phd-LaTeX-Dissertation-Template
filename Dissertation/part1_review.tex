\chapter{Обзор} \label{chapt1}

\section{Обзорнейший обзор} \label{sect1_1}

The CoNLL--SIGMORPHON 2018 Shared Task: Universal Morphological Reinflection

https://arxiv.org/pdf/1810.07125.pdf

Для 103 языков были даны леммы и морфологические характеристики, 
нужно было получить правильную словоформу -- это первая задача. 
Вторая -- дан фрагмент текста, нужно для слова указать его морфологические 
характеристики (и лемму?).

Мы можем сделать \textbf{жирный текст} и \textit{курсив}.

Разработано программное обеспечение wcorpus~\cite{vakbib_soft_wcorpus}.



\section{Открытые ресурсы по карельскому и вепсскому языку в интернете} \label{sect_open_krl_vep_inet}

В статье сотрудников Google~\cite{Prasad2018}  перечислены открытые ресурсы для языков мира. 
Посмотреть эти ресурсы и перечислить - где и сколько есть текстов и статей 
для карельского и вепсского языков. Сравнить (в процентах) с тем, что есть 
в электронном виде в ВепКаре, в бумажном виде в ИЯЛИ.

%@article{prasad2018mining,
%  title={Mining Training Data for Language Modeling Across the World's Languages},
%  author={Prasad, Manasa and Breiner, Theresa and van Esch, Daan},
%  year={2018}
%}


\section{Обзор компьютерных программ для морфологической обработки}

\subsection{Финский язык} \label{sect1_2}

Статья о лемматизаторе FinnPos~\cite{silfverberg2016finnpos}.

%https://github.com/mpsilfve/FinnPos

\section{Europarl Corpus and Words2Grids}

В работе~\cite{Fam2018tools} различают две структуры: \emph{Paradigm tables} 
и \emph{Analogical grids}. 

На языке Python разработана программа Words2Grids, которая по списку словоформ 
создаёт Analogical grids, которые, в свою очередь, нужны для создания 
таблиц склонений, то есть Paradigm tables.

Эксперименты проводились на 11 языках в корпусе текстов Europarl. 
Особняком стоят результаты по финскому (агглюнативному) языку (см. табл. 1 на с. 1063).



