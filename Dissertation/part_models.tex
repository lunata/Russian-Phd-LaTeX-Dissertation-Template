\chapter{Модели} \label{chapt_models}

\section{Введение} \label{sect_model_intro}
Академик Ю.~Д.~Апресян еще в 1981 году писал о необходимости проведения широких теоретических поисковых исследований в области формальных моделей языка~\cite{Apresjan1981}.

В последние годы под языковыми моделями понимают...~\cite{Petroni2019}. 

В диссертационной работе используются модели другого рода.



\section{Система правил словоизменения} \label{sect_rules_flextion}

Глава 1.(с. 20-25). Морфология на основе слов. Есть набор словоформ. 
Перебираем все варианты перехода 
от одной формы к другой, n*(n-1)/2 правил~\cite[20--25]{Albright2002stem}.

Чтобы ограничить чрезмерное количество правил, предлагаем и применяем эвристики:

Эвристика 1: пусть существет только одна базовая словоформа 
(остальные получаем с помощью правил из базовой). 
Тогда для n словоформ в парадигме будет n правил~\cite{Albright2002stem}.

Эвристика 2: пусть существет две базовых словоформы 
(остальные получаем с помощью правил из двух базовых). 
Тогда для n словоформ в парадигме будет ??? правил.

Прекрасная визуализация морфологических правил в виде графа 
(рис. на с. 21-22)~\cite[21--22]{Albright2002stem}: 
вершина~--- это словоформа, ребро~--- это правило.

Построим такую визуализацию для морфологических правил карельского языка. 




\section{Частотный анализ} \label{sect_freq}


Вот курсовик с анализом частоты букв (слогов?) для эрзянского языка.

На странице 6 (последнее предложение)
\url{https://nauchkor.ru/pubs/chastotnyy-analiz-finno-ugorskih-yazykov-rossii-5697aa605f1be742640000df#page=6}
мысль о том, чтобы подсчитать частотность слов в корпусе и указывать эту информацию в словаре.

Там же, страница 8. Подсчитать в корпусе ВепКар частоту символов для каждого из языков, 
подсчитать для русского в wcorpus. Сравнить.

% Вывод
\section{Выводы по главе \ref{chapt_models}} \label{sect_conclusion_models}

 

