\chapter{Модели} \label{chapt_models}

\section{Введение} \label{sect_model_intro}
Академик Ю.Д. Апресян еще в 1981 году писал о необходимости проведения широких теоретических поисковых исследований в области формальных моделей языка~\cite{Apresjan1981}.

\section{Частотный анализ} \label{sect_ideas_1}


Вот курсовик с анализом частоты букв (слогов?) для эрзянского языка.

На странице 6 (последнее предложение)
\url{https://nauchkor.ru/pubs/chastotnyy-analiz-finno-ugorskih-yazykov-rossii-5697aa605f1be742640000df#page=6}
мысль о том, чтобы подсчитать частотность слов в корпусе и указывать эту информацию в словаре.

Там же, страница 8. Подсчитать в корпусе ВепКар частоту символов для каждого из языков, 
подсчитать для русского в wcorpus. Сравнить.


