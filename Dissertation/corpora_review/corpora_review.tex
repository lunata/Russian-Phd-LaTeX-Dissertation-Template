 \subsection{Примеры лингвистических корпусов}

 \subsection{Europarl Corpus and Words2Grids}

В работе~\cite{Fam2018tools} различают две структуры: \emph{Paradigm tables} 
и \emph{Analogical grids}. 

На языке Python разработана программа Words2Grids, которая по списку словоформ 
создаёт Analogical grids, которые, в свою очередь, нужны для создания 
таблиц склонений, то есть Paradigm tables.

Эксперименты проводились на 11 языках в корпусе текстов Europarl. 
Особняком стоят результаты по финскому (агглюнативному) языку (см. табл. 1 на с. 1063).


\subsection{Морфологическая разметка ГИКРЯ}

Читать и писать о работе~\cite{Selegey2016}...

\subsection{Томский диалектный корпус}
Очень детальная семантическая разметка по жанрам и тематикам проведена в Томском диалектном корпусе~\cite{Zemicheva2019}.
Томский  диалектный  корпус\footnote{ См. http://losl.tsu.ru/corpus/demo}  создаётся  с  2017  г.  на  материале  диалектологических  экспедиций  в  среднеобский  регион  (Томская,  центральная  часть  Кемеровской  области).  Объём  ресурса на  апрель 2020 года более 1.7 млн словоупотреблений. Корпус имеет большой временной охват (70 лет) и детальную тематическая разметка (73  темы). 