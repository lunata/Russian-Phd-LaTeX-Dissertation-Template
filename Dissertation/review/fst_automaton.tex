\section{Конечные преобразователи и автоматы} \label{sect_review_automaton}

Математические основы конечных автоматов и трансдьюсеров были разработаны 
несколько десятилетий назад~\cite{MohriChapter4Lothaire2005applied}.
%Lothaire2005applied}.


Вилфред Брауэр. Введение в теорию конечных автоматов. М.: Радио и Связь 392 с; 
1987 г. \todo{Прочитать (см. отечеств. терминологию, в библиотеке или libex).}

%Разработано программное обеспечение wcorpus~\cite{vakbib_soft_wcorpus}.


\subsection{Взвешенные трансдьюсеры} \label{sect_weighted_transducers}

Трансдьюсеры могут использоваться для отображения и связывания разных видов данных, 
например, слова и последовательности фонем. 
Подобные автоматы нужны при разработке систем распознавания речи~\cite[с.~200]{MohriChapter4Lothaire2005applied}.

Веса позволяют указать в такой модели отображения наличие неопределённости. 
Например, во взвешенных трансдьюсерах одному слову могут отвечать несколько вариантов 
произношения слова с разными рангами или вероятностями~\cite[с.~200]{MohriChapter4Lothaire2005applied}.


Базовые положения о морфологических трансдьюсерах в статье "Carlson, L., 2005. Inducing a morphological transducer from inflectional paradigms. Inquiries into Words, Constraints and Contexts, p.18-24."
%см. /data/all/docs/science/linguistics/finite-State-Transducers/about/morpho_from_paradigm_2005_Lauri_Carlson_42-48.pdf


История вопроса о конечных преобразователях с 1980 по 2005 год.
\todo{Karttunen, L. and Beesley, K.R., 2005. Twenty-five years of finite-state morphology. Inquiries Into Words, a Festschrift for Kimmo Koskenniemi on his 60th Birthday, pp.71-83.}
% 25years_FSMorphology_KarttunenBeesley2005.pdf

И в целом история вычислительной лингвистики \todo{Karttunen, L., 2007. Word play. Computational Linguistics, 33(4), pp.443-467.}
% Word_play_2007Karttunen.pdf

% As many observers have indicated, the most promising approaches will probably integrate rule-based and corpus-based methods. 
Как отмечают многие наблюдатели, наиболее многообещающие подходы, вероятно, будут включать методы, основанные на правилах и корпусе~\cite{Hutchins1999}.



%\section{Software automaton} \label{sect_automaton_soft}

%В статье сотрудников Google~\cite{Prasad2018}  перечислены открытые ресурсы для языков мира. 

%и \emph{Analogical grids}. 



