\section{Корпусная лингвистика}\label{sect_review_corpus_linguistics}

Корпусная лингвистика~--- это раздел компьютерной (прикладной) лингвистики, 
описывающий принципы и методы 
построения лингвистических корпусов (корпусов текстов)
и методы использования корпусных данных~\cite[3]{Zakharov2005},~\cite[407]{Kibrik2019}.
Другими словами, корпусной лингвистикой называют подход к изучению языка с использованием компьютеров для анализа большого количества языковых данных, как письменных, так и речевых. Такие данные называют корпусами~\cite[11]{BrezinaMcEnery2020}.

Существует множество определений понятия <<корпус текстов>>. Например, из учебника Эдварда Финегана: репрезентативная коллекция текстов, обычно в машиночитаемой форме, включающая информацию о ситуации, в которой возник каждый текст, например о докладчике или авторе, адресате или аудитории,~\cite[24]{Finegan2006}. В.П. Захаров определяет корпус текстов как <<большой, представленный в машиночитаемом формате, унифицированный, структурированный, размеченный, филологически компетентный массив языковых данных, предназначенный для решения конкретных лингвистических задач>>~\cite[11]{ZakharovBogdanova2020}.
Т. Мак-Энери и др. определяют корпус как коллекцию машиночитаемых аутентичных текстов (включая транскрипцию разговоров), специально отобранных для представления определенного языка~\cite[5]{McEnery2006}.


Из всех определений можно сделать вывод, что современный корпус текстов должен обладать: 
    1. целью (корпус формируется для решения конкретных лингвистических задач),
    2. машиночитаемым форматом, 
    3. репрезентативностью (как результат особой процедуры отбора текстов), 
    4. наличием металингвистической информации~\cite[11-12]{ZakharovBogdanova2020}.

Корпус можно рассматривать как \emph{метод поиска} 
на больших массивах текстов~\cite[18]{Kozera2019CorpusAsMethod}. 
С помощью корпусов можно оценить частотность языковых конструкций, 
выявить образцы сочетаемости слов~\cite[18]{Kozera2019CorpusAsMethod}. 


Научное значение корпусов заключается в том, что наличие корпуса обеспечивает воспроизводимость, 
возможность повторить эксперимент~\cite[409]{Kibrik2019} 
хотя бы в пределах одного корпуса\footnote{%
    Рассчитывать на воспроизводимость эксперимента в разных корпусах можно 
    при соблюдении требования \emph{репрезентативности}, см. следующий раздел.
}. 
Трудность здесь может крыться в том, что <<живые>> корпусы, 
то есть те, над которыми продолжают работать исследователи, 
постоянно пополняются новыми текстами, увеличивается объём разметки. 
Этот рост корпуса может менять результаты эксперимента. 
Ситуацию здесь может спасти то, 
что в достаточно больших корпусах добавление новых данных будет небольшим 
относительно объёма данных всего корпуса\footnote{%
    %
    % Объём и в процентах новых слов в ВепКар 
    Приведём обратный пример, пример о добавлении большого объёма новых данных 
    в относительно небольшой корпус ВепКар. 
    Были обработаны и преобразованы в машиночаемую форму словарные статьи 
    Сопоставительно-ономасиологического словаря диалектов карельского, вепсского, саамского языков (кратко, словарь СОСД)~\cite{SOSD2007}. 
    %одержащего 1500 понятий 
    %\TODO{TODO можно ли чуть подробнее о СОСД? Сколько типов связей и сколько связей он содержит? Сколько диалектов? Чем он грандиозен?} 
    До включения данных словаря СОСД система ВепКар содержала 35~098 лемм, 800 переводов.
    После обработки СОСД в ВепКар было включено 1425 понятий 
        (\TODO{одно понятие через значение связывает несколько десятков диалектов}), 
    20 тыс. лемм, связанных с понятиями, из них 16 тыс. новых (\TODO{то есть число лемм выросло на 46\% от 35 тыс.}); 
    130 тыс. переводов (связей между значениями лемм из разных языков, наречий). 
    При этом было создано 60 тыс. связей между значениями лемм и словами из текста, 
    50 тыс. слов в текстах получили новые связи. Но только 2 тыс. слов до этого не имели разметки вообще.
    Эти цифры говорят и о том, что Сопоставительно-ономасиологический словарь был грандиозным проектом, 
    содержащим результаты работы большого коллектива учёных, и о том,
    что корпус ВепКар ещё далёк от насыщения по числу слов и текстов.
}.


