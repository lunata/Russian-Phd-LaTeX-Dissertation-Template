\section{Компьютерная морфология}\label{sect_review_comp_morphology}

Морфология -- это раздел лингвистики, изучающий структуру слова и его грамматические значения~\cite{MitreninaNikolaevLando2016}. Другими словами, морфология изучает
1) часть речи,
2) словоизменение,
3) словообразование,
4) грамматическое значение (что слово означает в предложении). 

Компьютерная морфология анализирует и синтезирует слова программными средствами~\cite{MitreninaNikolaevLando2016}. 

В компьютерной морфологии взаимосвязаны три базовых понятия: лемма, граммема и словоформа.
\begin{enumerate}
    \item \emph{Лемма}~-- это базовая, каноническая форма слова. 
        Например, инфинитив у глагола.

    \item Грамматическое значение представляется в виде набора \emph{граммем}. 
        Граммему также называют грамматической характеристикой, 
          морфологическим признаком, морфологическим свойством, 
          тегом (при разметке текста). 
        Граммемы группируются по категориям (падеж, время и т.~д.). 
        Одна и та же форма слова не может иметь две граммемы одной категории 
        (например, глагол не может иметь сразу форму прошлого и настоящего времени). 
        С другой стороны, формы могут совпадать, и их нужно уметь различать. 

    \item \emph{Словоформа}~-- это... \TODO{TODO}
\end{enumerate}


\subsubsection{Морфологическое словоизменение и морфологический анализ}

Морфологическое словоизменение\footnote{%
    \emph{Морфологическое словоизменение} по-английски звучит как
    ``inflectional morphology''. 
    \emph{Система} и \emph{задача морфологического словоизменения} 
    будут переводиться как 
    ``morphological inflection system'' и ``morphological inflection task'' соответственно, 
    см. например, американскую статью~\cite{King2020seq2seqRussianMA}.
} или просто словоизменение~--
это отображение лемм и набора морфологических признаков 
на соответствующую словоформу~\cite[2821]{Cruz-Anastasopoulos-Stump2020Chatino}.

Под \emph{морфологическим анализом} слова или словоформы (morphological analysis) 
подразумевается определение леммы и  
грамматических характеристик словоформы~\cite{MitreninaNikolaevLando2016}.

Таким образом, если задачу <<морфологического словоизменения>> обозначить как прямую  
(по лемме и признаку нужно найти словоформу), 
то <<морфологический анализ>> (по словоформе требуется найти 
лемму и морфологические свойства) будет обратной задачей. 

\TODO{Todo: Добавить сюда адаптацию для нашего языка рисунка из статьи Гарри ``Fine-grained Morphosyntactic Analysis and Generation Tools for More Than One Thousand Languages''}

Отметим, что в ряде языков отсутствует морфологическое словоизменение, 
например, в языках йоруба и севернокитайском \TODO{(Vylomova et al., 2020, Todo ref: SIGMORPHON2020, page 2)}.




\subsubsection{Paradigm Cell Filling Problem}

С задачей морфологического анализа тесно связана задача 
``a paradigm cell filling problem'' (PCFP)~\cite{Ackerman08PartsAndWholes}.



\subsubsection{Соревнования по морфологическому анализу (и результаты?)}

Рассмотренные выше задачи решают на соревнованиях 
между компьютерными программами в рамках различных конференций:
\begin{description}[align=left]
    \item [Диалог]\footnote{См. \url{http://www.dialog-21.ru}}~-- 
            известная отечественная конференция 
            по компьютерной лингвистике.
            В 2019 году в преддверии этой конференции 
            было проведено соревнование LowResourceEval 
            по анализу нескольких языков России~\cite{Klyachko2019LowresourceEval}.
        \TODO{TODO коротенько о главных результатах.}

\item [SIGMORPHON]\footnote{См. \url{https://sigmorphon.github.io}}~-- 
                это группа учёных (одна из множества групп ACL\footnote{%
                ACL расшифровывается как ``Association for Computational Linguistics''.
                Ассоциация компьютерной лингвистики~-- это 
                международное научное и техническое сообщество специалистов 
                по обработке текстов и языковой информации.
            }), 
            занимающихся вычислительной морфологией и фонологией, 
            и одноимённая конференция, проводимая этими учёными.
            В соревновании по морфологии ``SIGMORPHON 2020'' 
            были использованы данные 
            разрабатываемого корпуса ВепКар~\cite{Vylomova2020SIGMORPHON}.

            \TODO{TODO коротенько о главных результатах.}
\end{description}



\subsection{Вепсский и карельский языки}\label{sect_review_veps_karelian}

Вепсский и карельский языки являются языками с богатой морфологией. \TODO{TODO: привести число словоформ в парадигме именной и глагольной формы языков. В виде таблицы? Ссылка на нашу статью?}

Моделирование грамматических функций (modeling of the grammatical functions) 
для языков с богатой морфологией (rich morphology) крайне важно. 
Это особенно трудная и важная задача 
для малоресурсных языков~\cite[2820]{Cruz-Anastasopoulos-Stump2020Chatino}.






\subsubsection{Открытые ресурсы по карельскому и вепсскому языку в интернете} \label{sect_open_krl_vep_inet}

В статье сотрудников Google~\cite{Prasad2018} перечислены открытые ресурсы для языков мира. 
Посмотреть эти ресурсы и перечислить - где и сколько есть текстов и статей 
для карельского и вепсского языков. Сравнить (в процентах) с тем, что есть 
в электронном виде в ВепКаре, в бумажном виде в ИЯЛИ.

%@article{prasad2018mining,
%  title={Mining Training Data for Language Modeling Across the World's Languages},
%  author={Prasad, Manasa and Breiner, Theresa and van Esch, Daan},
%  year={2018}
%}







\subsection{Обзор компьютерных программ для морфологической обработки}

\subsection{Финский язык}\label{sect_review_fin}

Статья о лемматизаторе FinnPos~\cite{silfverberg2016finnpos}.

%https://github.com/mpsilfve/FinnPos
