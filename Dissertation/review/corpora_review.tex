 \subsection{Примеры лингвистических корпусов}

Бывают разные подходы к разработке корпусов\ldots 

В мире существуют сотни больших лингвистических корпусов, например, Национальный корпус русского языка\footnote{См. \url{https://ruscorpora.ru}}, Британский национальный корпус\footnote{См. \url{http://www.natcorp.ox.ac.uk}}, 
Чешский национальный корпус\footnote{См. \url{https://www.korpus.cz}}. 

Для долговременной работы над корпусом нужен научный коллектив, включение корпусных исследований в план научной работы. Например, для работ над Чешским национальным корпусом был создан одноимённый институт, над созданием и пополнением Национального корпуса русского языка работают сотрудники нескольких университетов и институтов РАН. 

Хотя есть обратный пример веб-корпусов уральских языков, построенных практически в одиночку Т. Архангельским. Разница здесь в том, что веб-корпуса строятся с помощью автоматической обработки текстов сети Интернет и содержат только автоматическую разметку. Большие коллективы лингвистов нужны, чтобы выполнять тонкую настройку, то есть делать ручную разметку, создавать так называемый «золотой стандарт» или размеченную вручную часть корпуса, которая в дальнейшем будет использоваться в различных экспериментах. Примером может служить глубоко аннотированный корпус текстов русского языка СинТагРус, где каждое слово в тексте привязано к какой-либо словарной статье комбинаторного словаря \cite{Inshakova2019}.


\subsubsection{Британский национальный корпус}

\subsubsection{Чешский национальный корпус}

\subsubsection{Национальный корпус русского языка}

\subsubsection{СинТагРус}

 \subsubsection{Europarl Corpus and Words2Grids}

В работе~\cite{Fam2018tools} различают две структуры: \emph{Paradigm tables} 
и \emph{Analogical grids}. 

На языке Python разработана программа Words2Grids, которая по списку словоформ 
создаёт Analogical grids, которые, в свою очередь, нужны для создания 
таблиц склонений, то есть Paradigm tables.

Эксперименты проводились на 11 языках в корпусе текстов Europarl. 
Особняком стоят результаты по финскому (агглюнативному) языку (см. табл. 1 на с. 1063).


\subsubsection{Морфологическая разметка ГИКРЯ}

Читать и писать о работе~\cite{Selegey2016}...

\subsubsection{Корпуса уральских языков Поволжья}
Т. Архангельским разработаны веб-корпусы для пяти уральских языков\footnote{См. \url{http://volgakama.web-corpora.net}}: коми-зырянский, луговой марийский, мокшанский, удмуртский, эрзянский~\cite{Arkhangelskiy2020}. Удмуртский корпус разрабатывался в соавторстве с М. Медведевой, данные для звуковой части удмуртского корпуса собрала Е. Георгиева. Корпусы также включают тексты, извлечённые из публичных записей в соцсетях (в основном ВКонтакте), поэтому представленный язык близок к разговорной речи. Для каждого языка был разработан морфологический анализатор на основе правил. Анализаторы работают по данным словарей, поэтому в корпусах не распознаны несловарные слова или слова с опечатками. Проанализировано от 80\% до 96\% слов в корпусах. В основном контекст при морфологическом анализе не учитывается и анализатор выдаёт все возможные леммы для данной словоформы~\cite[58--59]{Arkhangelskiy2020}. Для малоресурсных языков важно, чтобы при автоматическом морфологическом анализе сохранялись в разметке все возможные формы слова для последующей проверки и выбора правильной формы лингвистом~\cite[61]{Arkhangelskiy2020}. Архангельский Т. называет такой корпусный менеджер в корпусе «дружественным к неоднозначности».

Жанровое разнообразие этих пяти корпусов таково. Основную массу составляют тексты электронных газет и журналов. Также корпусы включают художественную литературу, научные статьи, переводы Библии, статьи Википедии и официальные тексты. Большинство текстов написаны в 2010--2019 годах~\cite[59]{Arkhangelskiy2020}.

\subsubsection{Томский диалектный корпус}
Очень детальная семантическая разметка по жанрам и тематикам проведена в Томском диалектном корпусе~\cite{Zemicheva2019}.
Томский  диалектный  корпус\footnote{ См. http://losl.tsu.ru/corpus/demo}  создаётся  с  2017  г.  на  материале  диалектологических  экспедиций  в  среднеобский  регион  (Томская,  центральная  часть  Кемеровской  области).  Объём  ресурса на  апрель 2020 года более 1.7 млн словоупотреблений. Корпус имеет большой временной охват (70 лет) и детальную тематическая разметка (73  темы). 
