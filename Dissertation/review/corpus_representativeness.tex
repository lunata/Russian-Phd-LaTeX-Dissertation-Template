
\subsection{Полнота, репрезентативность корпуса, достоверность корпусных исследований} \label{sect_corpus_representativeness}

Для исследователей, работающих с корпусом, важен факт, что 
явление, не встретившиеся в корпусе, не получит отражения в описании. 
Поэтому 
<<как бы ни было редко (исследуемое явление) в языке, 
хотелось бы чтобы оно нашло отражение в корпусе>>~\cite[с.~413]{Kibrik2019}. 
Таким образом, \emph{требование полноты корпуса}~--- это включение таких текстов, 
чтобы исследуемое явление было представлено в корпусе 
(и желательно во всём своём многообразии).


\emph{Репрезентативность корпуса} 
(не путать с представительностью)\footnote{Представительный корпус -- 
это корпус, обеспечивающий максимально широкое покрытие  
различных типов текстов и функциональных стилей~\cite{Sharov2004}.
} 
(текстовых примеров) 
эмпирически определяют тем, в какой степени эти примеры показывают 
вариативность исследуемого явления~\cite{Biber1993representativeness}. 
%
\emph{Требование репрезентативности}~--- это выбор такого подмножества текстов 
для включения в корпус, 
чтобы выбранные тексты 
<<отражали те или иные параметры исследуемого языкового явления в той же пропорции, 
что и в языке вообще или в некотором исследуемом подмножестве языка>>~\cite[с.~413]{Kibrik2019}.

Из этих требований и определений следует, что 
нельзя говорить об универсальной полноте или репрезентативности корпуса, 
но можно говорить 
о степени репрезентативности корпуса при решении конкретной лингвистической задачи.



В работе~\cite{Belikov2013}
оценивается достоверность корпусных исследований. 
Часто исследователи подвержены соблазну распространить опыт, полученный на конкретном корпусе, 
на весь язык, что неправомерно~\cite{Belikov2013}.

Каковы границы применимости разрабатываемого Открытого корпуса вепсского и карельского языков? 
Для ответа на этот вопрос нужно определить, тексты каких жанров и в какой пропорции включены в корпус ВепКар. 
Объём корпуса ВепКар составляет соответственно 3~007 текстов и 1~084~679 слов в текстах\footnote{ Данные на 10 февраля 2021~г. Cм. подробнее 
\href{http://dictorpus.krc.karelia.ru/ru/stats/by\_corp}{http://dictorpus.krc.karelia.ru/ru/stats/by\_corp}.}.

На рисунках~\ref{fig:text_distr_by_corpus} и \ref{fig:text_distr_by_genre} 
показано распределение текстов по подкорпусам и жанрам для вепсского языка 
и наречий карельского языка\footnote{Данные 
                                     на 23 апреля 2020~года. Cм. подробнее 
			\href{http://dictorpus.krc.karelia.ru/ru/corpus/corpus}{http://dictorpus.krc.karelia.ru/ru/corpus/corpus}.}.

\begin{figure}
    \centering
    \includegraphics[width=1.0\textwidth,keepaspectratio=true]{text_distr_by_corpus.png}
    \caption[Распределение текстов по подкорпусам]{Распределение текстов по подкорпусам (вепсский язык и наречия карельского языка).}
    \label{fig:text_distr_by_corpus}
\end{figure}
%\bigskip

\begin{figure}
    \centering
    \includegraphics[width=1.0\textwidth,keepaspectratio=true]{text_distr_by_genre.png}
    \caption[Распределение текстов по жанрам]{Распределение текстов по жанрам (вепсский язык и наречия карельского языка).}
    \label{fig:text_distr_by_genre}
\end{figure}

На рис.~\ref{fig:text_distribution_by_date} показана гистограмма 
с числом текстов корпуса ВепКар по годам с 1918 по 2018 год. 
2041~текст (86,67\% или $6/7$ от всего числа текстов) не имеют информации о дате записи.
\begin{figure}
    \centering
    \includegraphics[width=1.0\textwidth,keepaspectratio=true]{text_distribution_by_date.png}
    \caption{Распределение числа текстов по годам.}
    \label{fig:text_distribution_by_date}
\end{figure}

Своевременно ли говорить о сбалансированности и представительности корпуса ВепКар? Вопрос остаётся открытым, нужны дополнительные исследования.
%Ответом может послужить информация о доле слов из словаря, употреблённых в текстах корпуса. 
%\nata{TODO: Подсчитать долю / процент слов (для каждого из языков), которые встречаются в текстах. Процент + абсолютное число слов словаря в текстах.}
%(Добавить эту статистику в корпус?)




