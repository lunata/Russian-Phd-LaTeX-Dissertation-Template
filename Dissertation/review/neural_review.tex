\section{Нейронные сети} \label{sect_nn}

\subsection{Введение} \label{sect_nn_review}

Глубинные нейронные сети -- это рычаг, который, опираясь на большие массивы данных, сдвинул многие камешки-задачи в лингвистике~\cite[2827]{Cruz-Anastasopoulos-Stump2020Chatino}.

Посимвольное представление вместо пословного 
даёт преимущество при обучении нейронных сетей 
различению морфологических свойств, 
особенно при анализе редких и новых слов~\cite[868]{Belinkov2017NeuralLearnMorphology}.

При решении самых разных лингвистических задач, 
в том числе при определении части речи слова (POS tagging), 
используют \emph{векторное представление слов}\footnote{%
    Word embeddings (векторное представление слов или языковая модель)~--- 
    это представление слов в виде массива чисел (n-мерного вектора), 
    построенного с помощью 
    дистрибутивно-семантической модели 
    (distributional semantic model или DSM)~\cite[1]{SurveyDSM2018Bakarov}.
} 
и нейронные сети.


Как оценить качество построенного векторного представления? 
То есть качество построенной дистрибутивно-семантической модели. 
Этот вопрос является открытым. Методы оценки делят на~\cite{SurveyDSM2018Bakarov}:

 \emph{внешние методы оценки} (extrinsic evaluation).
 Здесь оценка внешняя относительно языковой модели. 
 Оценивается результат решения задачи, в которой используется модель. 
 Например, в POS tagging оценивается точность указания части речи.


 \emph{внутренние методы оценки}~--- \TODO{TODO: дочитать и дописать}

