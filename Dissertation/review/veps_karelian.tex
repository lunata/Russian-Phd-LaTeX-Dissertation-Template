\newpage


\subsection{Вепсский и карельский языки}\label{sect_review_veps_karelian}

Вепсский и карельский являются агглютинативными языками с элементами флективности. 
Все словоизменительные категории в них образуются путем механического присоединения 
к основе одного или нескольких аффиксов, выражающих различные грамматические 
категории (например, падеж и число у именных частей речи; наклонение, время, лицо и число у глаголов). 

В процессе анализа словоизменительной системы вепсского и карельского языков 
важно учитывать наличие одноосновных и двуосновных имен и глаголов~\cite[57]{KrlNovak2019PeoplesOfKarelia}.
\TODO{TODO: добавить ссылку для этого утверждения с номерами страниц из книги <<Вепсский глагол>> \cite{ZaitsevaNG2002VepsVerb}.}
Таким образом, в этих языках для построения словоизменительных форм имён и глаголов 
нужно знать одну или две основы слова. 
%
<<Словоформа далеко не всегда содержит в себе однозначное указание на внешний вид гласной (слабой  гласной)  основы\ldots, а также на отсутствие или наличие у лексемы согласной основы\ldots, 
автоматическое их заполнение невозможно>> --- указано в статье 
о правилах \emph{генерации словоформ}%
%
\footnote{Генерация словоформ~--- это одна из четырёх морфологических операций, 
см. рис.~\ref{fig:inflectional_operations}.} % 
%
именных частей речи 
для новописьменных вариантов карельского языка~\cite[684]{rulesNominalKrl2020NovakUgricStudies}.
%
Пока не для всех случаев существуют формализованные (алгоритмические) способы определения основ. 
В этих трудных случаях следует обращаться к словарям, в которых указаны эти основы 
(например, <<Орфографический словарь вепсского языка>>~\cite{ZaitsevaNG2012OrphDict}, 
<<Большой карельско-русский словарь (ливвиковское наречие)>>~\cite{Boiko2019livvi}).

Отметим, что у людиковского и ливвиковского наречий карельского языка 
возвратные\footnote{Возвратные глаголы выражают действие, направленное на само действующее лицо.}
формы глаголов имеют свою парадигму, 
то есть склоняются по другим правилам, чем обычные (невозвратные) глаголы~\cite[268]{Novak2019Grammar}. 


\newpage

\subsubsection{О богатой морфологии, морфологические анализаторы вепсского и карельского языков, результаты SIGMORPHON 2019 (чатино и вепсский язык)}

Вепсский и карельский языки являются языками с богатой морфологией.
\TODO{TODO: привести число словоформ в парадигме именной и глагольной формы языков. В виде таблицы? Ссылка на нашу статью?}

Моделирование грамматических функций (modeling of the grammatical functions)
для языков с богатой морфологией (rich morphology
\TODO{TODO: Дать определение <<богатой морфологии>>}
) крайне важно.
Это особенно трудная и важная задача
для малоресурсных языков~\cite[2820]{Cruz-Anastasopoulos-Stump2020Chatino}.


%Let us describe several works devoted to the development of morphological analyzers for the Veps and Karelian languages.
Опишем несколько работ, посвящённых разработке морфологических анализаторов вепсского и карельского языков.
\begin{itemize}
%  \item The Giellatekno language research group is mainly engaged in low-resource languages, the project covers about 50 languages~\cite{Moshagen2014}. Our project has something in common with the work of Giellatekno in that (1) we work with low-resource languages, (2) we develop software and data with open licenses.
  \item Основным предметом исследования языковой группы Giellatekno 
      являются малоресурсные языки.  
        Разработаны лингвистические ресурсы 
        для порядка 50 языков~\cite{Moshagen2014}. 
        Общее в нашей работе с исследователям из Giellatekno в том, что 
        (1) мы также работаем с малоресурсными языками, 
        (2) мы разрабатываем программное обеспечение 
        и публикуем словари и корпуса текстов с открытой лицензией.
      Ключевую роль в разрабатываемых в Giellatekno языковых технологиях 
        играют формальные подходы. 
        В Giellatekno работают с морфологически богатыми языками.
        Для анализа и генерации словоформ (морфологический синтез и анализ) 
        они используют конечные преобразователи 
        (FST или finite-state transducers)~\cite{Moshagen2014}. 
% 
%  \item There is a texts and words processing library for the Uralic languages called UralicNLP~\cite{UralicNLP2019Hamalainen}.
%  This Python library provides interface to such Giellatekno tools as FST for processing morphology and constraint grammar for syntax. 
%  The UralicNLP library lemmatizes words in 30 Finno-Ugric languages and dialects including the Livvi dialect of the Karelian language (\textit{olo} -- language code).
  \item Для обработки слов и текстов на уральских языках разработана 
      библиотека UralicNLP~\cite{UralicNLP2019Hamalainen}. 
        Эта библиотека на языка Python предоставляет интерфейс 
        к конечным преобразователям Giellatekno 
        и к грамматикам ограничений (constraint grammar) той же системы. 
        Программный код библиотеки доступен 
        онлайн\footnote{См. \url{https://github.com/mikahama/uralicNLP}}.
        Библиотека UralicNLP выполняет лемматизацию слов 
        для 30 финно-угорских языков, включая 
        ливвиковское наречие карельского языка (языковой код \textit{olo}).
\end{itemize}





\bigskip
По точности результаты по языку чатино значительно хуже
средних результатов соревнования ``SIGMORPHON 2019'' по 66 языкам.
Это косвенно указывает на сложность морфологии языка~\cite[2822]{Cruz-Anastasopoulos-Stump2020Chatino}.
\TODO{TODO: Данные о чатинском языке приведены по статье 2019 года McCarthy.
            Для вепсского взять те же данные из статьи~\cite{Vylomova2020SIGMORPHON}.}


\subsubsection{Открытые ресурсы по карельскому и вепсскому языку в интернете} \label{sect_open_krl_vep_inet}

В статье сотрудников Google~\cite{Prasad2018} перечислены открытые ресурсы для языков мира.
Посмотреть эти ресурсы и перечислить - где и сколько есть текстов и статей
для карельского и вепсского языков. Сравнить (в процентах) с тем, что есть
в электронном виде в ВепКаре, в бумажном виде в ИЯЛИ.

%@article{prasad2018mining,
%  title={Mining Training Data for Language Modeling Across the World's Languages},
%  author={Prasad, Manasa and Breiner, Theresa and van Esch, Daan},
%  year={2018}
%}








