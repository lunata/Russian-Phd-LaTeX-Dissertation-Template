\chapter{Комплекс компьютерных программ} \label{chapt_comp}

\section{Архитектура корпусного менеджера Dictorpus} \label{sect_arch_soft}


\section{Архитектура словаря Dictorpus} \label{sect_dictionary}

\subsection{Морфологический интерфейс к словарю} \label{sect_morpho_API_to_dict}

В статье~\cite[с.~65]{morphoAPI2dict2004Maxwell} утверждается, 
что для поиска слова (в словаре) нужен морфологический парсер, 
поскольку это даст возможность пользователю искать слово в словаре в любой форме. 
И тогда, даже не зная правил словоизменения, пользователь найдёт требуемую словарную статью (или список статей).

Специалист по языку может определить \emph{заглавное слово}\footnote{Заглавное слово~--- это слово, разъясняемое в словарной статье; слово, являющееся точкой входа в словарную статью, и то слово, по которому выполняется поиск в словаре. Это же слово мы называем \emph{леммой}.} в словаре по любой словоформе. У рядового пользователя словаря таких знаний может не быть.
При возрастании сложности морфологических правил меньше надежды, 
что пользователь словаря сможет верно угадывать лемму 
для поиска словарной статьи~\cite[с.~66]{morphoAPI2dict2004Maxwell}. 
Авторы статьи~\cite{morphoAPI2dict2004Maxwell} делают следующие выводы:
\begin{enumerate}
\item Для поиска в словаре можно использовать морфологический парсер, который будет по запросу пользователя определять лемму и находить требуемую статью в словаре.
\item Явное указание множества словоформ в словаре является проблематичным.
\end{enumerate}

\noindent
Второй вывод в статье не обоснован,
то есть отсутствует критика подхода: 

\begin{tabular}{|p{15cm}}
если известны правила словоизменения, то генерируются все словоформы и хранятся в словаре для последующего поиска по ним. 
\end{tabular}

\noindent
Это именно тот путь, который мы выбрали в словаре корпусного менеджера Dictorpus 
по следующим причинам. 

Во-первых, с публикации статьи 2004 года прошло 16 лет и современные компьютеры могут хранить в оперативной памяти все словоформы, как минимум для вепсского и карельского языков. У глагола (для одного диалекта) 115~словоформ в вепсском языке и 117~--- в карельском. У именных частей речи 45~словоформ в вепсском языке и 27~--- в карельском~\cite[с.~415--428]{ZaitsevaNG2012OrphDict} \cite{NovakPenttonenRuuskanenSiilin2019, Zaikov2000KarGram, ZaitsevaMI1981, ZaitsevaNG1981VepsName, ZaitsevaNG2002VepsVerb}. На~9~апреля 2020~г.\footnote{Актуальную ииформацию см. по адресу \url{http://dictorpus.krc.karelia.ru/ru/stats/by_dict}} в корпусе ВепКар 39~183 тысяч лемм и 754~197 словоформ на вепсском и карельском языке.

Во-вторых, прямая задача генерации словоформ по известным правилам и основам слова является более простой, чем обратная задача поиска леммы по словоформе.

Третья причина~--- скорость поиска~--- причиной не является, 
поскольку без проведения экспериментов нельзя сказать, что быстрее: 
поиск словоформ по базе данных или определение леммы и затем поиск леммы. 
Вероятно, разница во времени будет несущественна.

Таким образом, пользуясь терминологией статьи~\cite{morphoAPI2dict2004Maxwell}, 
мы утверждаем, что в корпусном менеджере Dictorpus 
реализован морфологический интерфейс к словарю, 
позволяющий находить словарные статьи как по лемме, так и по словоформе.




\section{Архитектура базы данных Dictorpus} \label{sect_arch_db}

