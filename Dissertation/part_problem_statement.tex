\chapter{Постановка задачи} \label{chapt_problem_formulation}

\section{Объект исследования} \label{sect_problem_obj}

Объект исследования~--- это зависимость между леммой, словоформами 
и набором грамматических признаков (грамсет).
Задачи морфологического синтеза и анализа решаются для слов на карельском языке.
Таким образом, объектом математического моделирования 
является естественный язык. 


\section{Предмет исследования} \label{sect_predmet_obj}

Предмет исследования~--- это морфологический синтез и морфологический анализ, 
в том числе задача лемматизации. 

Предмет исследования непосредственно соотнесен 
с целями исследования~\cite{Martishina2001Object}, 
с тем чтобы построить морфологический парсер и лемматизатор карельского языка. 

\section{Формализация задачи морфологического анализа} \label{sect_formal}

Придумать символы, отношения, правила, чтобы можно было формально показать работу 
разработанных правил словоизменения.

% Вывод
\section{Выводы по главе \ref{chapt_problem_formulation}} \label{sect_conclusion_problem_statement}

 
