\section{Компьютерная морфология} \label{sect_review_comp_morphology}

Морфология - это раздел лингвистики, изучающий структуру слова и его грамматические значения~\cite{MitreninaNikolaevLando2016}. Другими словами, морфология изучает
1) часть речи,
2) словоизменение,
3) словообразование,
4) грамматическое значение (что слово означает в предложении). 

Компьютерная морфология анализирует и синтезирует слова программными средствами~\cite{MitreninaNikolaevLando2016}. 

Грамматическое значение представляется в виде набора граммем. Граммемы группируются по категориям (падеж, время и т.д.). Одна и та же форма слова не может иметь две граммемы одной категории. С другой стороны формы могут совпадать, и их нужно уметь различать. Это~-- одна из задач морфологии.


Под \textbf{морфологическим анализом} подразумевается определение леммы (базовой, канонической вормы слова) и ее грамматических характеристик~\cite{MitreninaNikolaevLando2016}.
