\chapter{Обзор} \label{chapt_review}

\section{Обзорнейший обзор} \label{chapt_review_sect_intro}

The CoNLL--SIGMORPHON 2018 Shared Task: Universal Morphological Reinflection

https://arxiv.org/pdf/1810.07125.pdf

Для 103 языков были даны леммы и морфологические характеристики, 
нужно было получить правильную словоформу -- это первая задача. 
Вторая -- дан фрагмент текста, нужно для слова указать его морфологические 
характеристики (и лемму?).

Разработано программное обеспечение wcorpus~\cite{vakbib_soft_wcorpus}.

Разработана программа ``New written Tver Karelian dialects wordform generator''~\cite{vakbib_soft_Tver_generator}.



\section{Открытые ресурсы по карельскому и вепсскому языку в интернете} \label{sect_open_krl_vep_inet}

В статье сотрудников Google~\cite{Prasad2018} перечислены открытые ресурсы для языков мира. 
Посмотреть эти ресурсы и перечислить - где и сколько есть текстов и статей 
для карельского и вепсского языков. Сравнить (в процентах) с тем, что есть 
в электронном виде в ВепКаре, в бумажном виде в ИЯЛИ.

%@article{prasad2018mining,
%  title={Mining Training Data for Language Modeling Across the World's Languages},
%  author={Prasad, Manasa and Breiner, Theresa and van Esch, Daan},
%  year={2018}
%}


\section{Обзор компьютерных программ для морфологической обработки}

\subsection{Финский язык} \label{sect_review_fin}

Статья о лемматизаторе FinnPos~\cite{silfverberg2016finnpos}.

%https://github.com/mpsilfve/FinnPos

