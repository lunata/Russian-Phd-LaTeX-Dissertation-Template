\chapter{Оформление различных элементов} \label{chapt1}

\section{Обзорнейший обзор} \label{sect1_1}

The CoNLL--SIGMORPHON 2018 Shared Task: Universal Morphological Reinflection

https://arxiv.org/pdf/1810.07125.pdf

Для 103 языков были даны леммы и морфологические характеристики, 
нужно было получить правильную словоформу -- это первая задача. 
Вторая -- дан фрагмент текста, нужно для слова указать его морфологические 
характеристики (и лемму?).

Мы можем сделать \textbf{жирный текст} и \textit{курсив}.

Разработано программное обеспечение wcorpus~\cite{vakbib_soft_wcorpus}.



\section{Открытые ресурсы по карельскому и вепсскому языку в интернете} \label{sect_open_krl_vep_inet}

В статье сотрудников Google~\cite{Prasad2018}  перечислены открытые ресурсы для языков мира. 
Посмотреть эти ресурсы и перечислить - где и сколько есть текстов и статей 
для карельского и вепсского языков. Сравнить (в процентах) с тем, что есть 
в электронном виде в ВепКаре, в бумажном виде в ИЯЛИ.

%@article{prasad2018mining,
%  title={Mining Training Data for Language Modeling Across the World's Languages},
%  author={Prasad, Manasa and Breiner, Theresa and van Esch, Daan},
%  year={2018}
%}



\section{Форматирование текста} \label{sect1_2}

Мы можем сделать \textbf{жирный текст} и \textit{курсив}.

\section{Ссылки} \label{sect1_3}
Сошлёмся на библиографию.
Одна ссылка: \cite[с.~54]{Sokolov}\cite[с.~36]{Gaidaenko}.
Две ссылки: \cite{Sokolov,Gaidaenko}.
Много ссылок: %\cite[с.~54]{Lermontov,Management,Borozda} % такой «фокус» вызывает biblatex warning относительно опции sortcites, потому что неясно, к какому источнику относится уточнение о страницах, а bibtex об этой проблеме даже не предупреждает
\cite{Lermontov,Management,Borozda,Marketing,Constitution,FamilyCode,Gost.7.0.53,Razumovski,Lagkueva,Pokrovski,Sirotko,Lukina,Methodology,Encyclopedia,Nasirova,Berestova,Kriger}.
И~ещё немного ссылок:
\cite{Article,Book,Booklet,Conference,Inbook,Incollection,Manual,Mastersthesis,Misc,Phdthesis,Proceedings,Techreport,Unpublished}.
\cite{medvedev2006jelektronnye, CEAT:CEAT581, doi:10.1080/01932691.2010.513279,Gosele1999161,Li2007StressAnalysis, Shoji199895,test:eisner-sample,test:eisner-sample-shorted,AB_patent_Pomerantz_1968,iofis_patent1960}

%Попытка реализовать несколько ссылок на конкретные страницы для стандартной реализации:[\citenum{Sokolov}, с.~54; \citenum{Gaidaenko}, с.~36].

%Несколько источников мультицитата (только в biblatex)
%\cites[vii--x, 5, 7]{Sokolov}[v"--~x, 25, 526]{Gaidaenko} поехали дальше

Ссылки на собственные работы:~\cite{vakbib1, confbib1}

Сошлёмся на приложения: Приложение \ref{AppendixA}, Приложение \ref{AppendixB2}.

Сошлёмся на формулу: формула \eqref{eq:equation1}.

Сошлёмся на изображение: рисунок \ref{img:knuth}.

\section{Формулы} \label{sect1_4}

Благодаря пакету \textit{icomma}, \LaTeX~одинаково хорошо воспринимает
в~качестве десятичного разделителя и запятую ($3,1415$), и точку ($3.1415$).

\subsection{Ненумерованные одиночные формулы} \label{subsect1_3_1}

Вот так может выглядеть формула, которую необходимо вставить в~строку
по~тексту: $x \approx \sin x$ при $x \to 0$.

А вот так выглядит ненумерованая отдельностоящая формула c подстрочными
и надстрочными индексами:
\[
(x_1+x_2)^2 = x_1^2 + 2 x_1 x_2 + x_2^2
\]

При использовании дробей формулы могут получаться очень высокие:
\[
  \frac{1}{\sqrt{2}+
  \displaystyle\frac{1}{\sqrt{2}+
  \displaystyle\frac{1}{\sqrt{2}+\cdots}}}
\]

В формулах можно использовать греческие буквы:
\[
\alpha\beta\gamma\delta\epsilon\varepsilon\zeta\eta\theta\vartheta\iota\kappa%
\lambda\\mu\nu\xi\pi\varpi\rho\varrho\sigma\varsigma\tau\upsilon\phi\varphi%
\chi\psi\omega\Gamma\Delta\Theta\Lambda\Xi\Pi\Sigma\Upsilon\Phi\Psi\Omega
\]

\def\slantfrac#1#2{ \hspace{3pt}\!^{#1}\!\!\hspace{1pt}/
  \hspace{2pt}\!\!_{#2}\!\hspace{3pt}
} %Макрос для красивых дробей в строчку (например, 1/2)
Для красивых дробей (например, в индексах) можно добавить макрос
\verb+\slantfrac+ и писать $\slantfrac{1}{2}$ вместо $1/2$.

\subsection{Ненумерованные многострочные формулы} \label{subsect1_3_2}

Вот так можно написать две формулы, не нумеруя их, чтобы знаки <<равно>> были
строго друг под другом:
\begin{align}
  f_W & =  \min \left( 1, \max \left( 0, \frac{W_{soil} / W_{max}}{W_{crit}} \right)  \right), \nonumber \\
  f_T & =  \min \left( 1, \max \left( 0, \frac{T_s / T_{melt}}{T_{crit}} \right)  \right), \nonumber
\end{align}

Выровнять систему ещё и по переменной $ x $ можно, используя окружение
\verb|alignedat| из пакета \verb|amsmath|. Вот так:
\[
    |x| = \left\{
    \begin{alignedat}{2}
        &&x, \quad &\text{eсли } x\geqslant 0 \\
        &-&x, \quad & \text{eсли } x<0
    \end{alignedat}
    \right.
\]
Здесь первый амперсанд (в исходном \LaTeX\ описании формулы) означает
выравнивание по~левому краю, второй "--- по~$ x $, а~третий "--- по~слову
<<если>>. Команда \verb|\quad| делает большой горизонтальный пробел.

Ещё вариант:
\[
    |x|=
    \begin{cases}
    \phantom{-}x, \text{если } x \geqslant 0 \\
    -x, \text{если } x<0
    \end{cases}
\]

Кроме того, для  нумерованых формул \verb|alignedat| делает вертикальное
выравнивание номера формулы по центру формулы. Например, выравнивание
компонент вектора:
\begin{equation}
 \label{eq:2p3}
 \begin{alignedat}{2}
{\mathbf{N}}_{o1n}^{(j)} = \,{\sin} \phi\,n\!\left(n+1\right)
         {\sin}\theta\,
         \pi_n\!\left({\cos} \theta\right)
         \frac{
               z_n^{(j)}\!\left( \rho \right)
              }{\rho}\,
           &{\boldsymbol{\hat{\mathrm e}}}_{r}\,+   \\
+\,
{\sin} \phi\,
         \tau_n\!\left({\cos} \theta\right)
         \frac{
            \left[\rho z_n^{(j)}\!\left( \rho \right)\right]^{\prime}
              }{\rho}\,
            &{\boldsymbol{\hat{\mathrm e}}}_{\theta}\,+   \\
+\,
{\cos} \phi\,
         \pi_n\!\left({\cos} \theta\right)
         \frac{
            \left[\rho z_n^{(j)}\!\left( \rho \right)\right]^{\prime}
              }{\rho}\,
            &{\boldsymbol{\hat{\mathrm e}}}_{\phi}\:.
\end{alignedat}
\end{equation}

Ещё об отступах. Иногда для лучшей <<читаемости>> формул полезно
немного исправить стандартные интервалы \LaTeX\ с учётом логической
структуры самой формулы. Например в формуле~\ref{eq:2p3} добавлен
небольшой отступ \verb+\,+ между основными сомножителями, ниже
результат применения всех вариантов отступа:
\begin{align*}
\backslash! &\quad f(x) = x^2\! +3x\! +2 \\
  \mbox{по-умолчанию} &\quad f(x) = x^2+3x+2 \\
\backslash, &\quad f(x) = x^2\, +3x\, +2 \\
\backslash{:} &\quad f(x) = x^2\: +3x\: +2 \\
\backslash; &\quad f(x) = x^2\; +3x\; +2 \\
\backslash \mbox{space} &\quad f(x) = x^2\ +3x\ +2 \\
\backslash \mbox{quad} &\quad f(x) = x^2\quad +3x\quad +2 \\
\backslash \mbox{qquad} &\quad f(x) = x^2\qquad +3x\qquad +2
\end{align*}

Можно использовать разные математические алфавиты:
\begin{align}
\mathcal{ABCDEFGHIJKLMNOPQRSTUVWXYZ} \nonumber \\
\mathfrak{ABCDEFGHIJKLMNOPQRSTUVWXYZ} \nonumber \\
\mathbb{ABCDEFGHIJKLMNOPQRSTUVWXYZ} \nonumber
\end{align}

Посмотрим на систему уравнений на примере аттрактора Лоренца:

\[
\left\{
  \begin{array}{rl}
    \dot x = & \sigma (y-x) \\
    \dot y = & x (r - z) - y \\
    \dot z = & xy - bz
  \end{array}
\right.
\]

А для вёрстки матриц удобно использовать многоточия:
\[
\left(
  \begin{array}{ccc}
    a_{11} & \ldots & a_{1n} \\
    \vdots & \ddots & \vdots \\
    a_{n1} & \ldots & a_{nn} \\
  \end{array}
\right)
\]

\subsection{Нумерованные формулы} \label{subsect1_3_3}

А вот так пишется нумерованая формула:
\begin{equation}
  \label{eq:equation1}
  e = \lim_{n \to \infty} \left( 1+\frac{1}{n} \right) ^n
\end{equation}

Нумерованых формул может быть несколько:
\begin{equation}
  \label{eq:equation2}
  \lim_{n \to \infty} \sum_{k=1}^n \frac{1}{k^2} = \frac{\pi^2}{6}
\end{equation}

Впоследствии на формулы (\ref{eq:equation1}) и (\ref{eq:equation2}) можно ссылаться.

Сделать так, чтобы номер формулы стоял напротив средней строки, можно,
используя окружение \verb|multlined| (пакет \verb|mathtools|) вместо
\verb|multline| внутри окружения \verb|equation|. Вот так:
\begin{equation} % \tag{S} % tag - вписывает свой текст
  \label{eq:equation3}
    \begin{multlined}
        1+ 2+3+4+5+6+7+\dots + \\
        + 50+51+52+53+54+55+56+57 + \dots + \\
        + 96+97+98+99+100=5050 
    \end{multlined}
\end{equation}

Используя команду \verb|\labelcref| из пакета \verb|cleveref|, можно
красиво ссылаться сразу на несколько формул
(\labelcref{eq:equation1,eq:equation3,eq:equation2}), даже перепутав
порядок ссылок \verb|(\labelcref{eq:equation1,eq:equation3,eq:equation2})|.
