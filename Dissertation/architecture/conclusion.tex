\section{Выводы по главе \ref{chapt_complex_software}} \label{sect_conclusion_complex_software}

В главе описывается Открытый корпус вепсского и карельского языков (ВепКар), создаваемый лингвистами и программистами  Карельского научного центра РАН более 10 лет.  В 2009 году был создан «Корпус вепсского языка», в 2016 году корпус стал многоязычным: на базе вепсского был корпуса был создан карельский, включающий в себя три подкорпуса: собственно карельский, ливвиковский и людиковский.  Объединённый корпус получил название «Открытый корпус вепсского и карельского языков». Словари и тексты корпуса вместе с поисковой системой доступны онлайн\footnote{ См. \url{http://dictorpus.krc.karelia.ru}}.