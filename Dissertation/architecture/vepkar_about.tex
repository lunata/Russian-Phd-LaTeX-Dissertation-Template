
\section{Открытый корпус вепсского и карельского языков (ВепКар)} \label{sect_VepKar_about}

В этом подразделе дадим краткую справку о том, что из себя представляет 
разрабатываемый комплекс компьютерных программ -- лингвистический
корпус ВепКар. 

\todo{Тип корпуса по Захарову}

Что такое лингвистический корпус текстов? Это, в первую очередь, компьютерная программа для обработки и анализа текстов или это сами тексты? 

Ответом на эти вопросы могут послужить слова Никлауса Вирта, что: 
<<\ldots структура программы и строение данных неразрывно связаны между собой\ldots>>~\cite[с.~9]{Wirth1989AlgorithmsAndDataStructure}. 
Более того:   
<<\ldots данные предшествуют алгоритмам: нужно иметь некоторые объекты, 
прежде чем выполнять действия с ними\ldots>>~\cite[с.~8]{Wirth1985Algorithms+} 
-- утверждается в книге <<Алгоритмы + структуры данных = программы>>.
Мы можем развить эту мысль применительно к лингвистическим корпусам:
        
\textWithVerticalBar{15cm}{Корпусный менеджер + тексты = корпус.}

\noindent
При этом корпусный менеджер обслуживает как сами тексты, так и словарные статьи. 
В соответствии с этим положением о важности данных, далее будет  
сначала описана база данных, затем -- корпусный менеджер и словарь.
