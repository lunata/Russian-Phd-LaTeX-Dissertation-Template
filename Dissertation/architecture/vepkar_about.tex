
\section{Открытый корпус вепсского и карельского языков (ВепКар)} \label{sect_VepKar_about}

В этом подразделе дадим краткую справку о том, что из себя представляет 
лингвистический корпус ВепКар, в основе которого лежит 
разрабатываемый комплекс компьютерных программ.

В соответствии с классификацией, предложенной 
в учебнике по корпусной лингвистике~\cite[с.~12]{Zakharov2005}, 
разрабатываемый \emph{Открытый корпус вепсского и карельского языков}: 
\renewcommand{\outlinei}{itemize}
\renewcommand{\outlineii}{itemize}
\begin{outline}
    \1 является многоязычным корпусом;
    \1 является полнотекстовым корпусом в отличие, например, от НКРЯ\footnote{
        Национальный корпус русского языка (НКРЯ) \TODO{(TODO: найти статью о НКРЯ)} 
        выдаёт пользователям фрагменты текстов (2--3 предложения), 
        содержащие искомое слово, но полный текст не предоставляют. 
        Наверное, это связано с тем, что значительный пласт текстов корпуса написан в XIX-XX веках, 
        то есть права на них всё ещё принадлежат авторам и их наследникам. 
        Благодаря использованию открытой лицензии Creative Commons Attribution на тексты 
        и наличию разрешительных документов разработчики ВепКар не связаны таким ограничением. 
        Посему пользователи ВепКар могут работать с полными текстами без ограничений.
    };
    \1 относится ко всему языку, то есть включает все возможные жанры и стили;
    \1 включает разметку:
        \2[\textbullet] метатекстовую (название текста, дата создания, автор, жанр, место записи и др.);
        \2[\textbullet] морфологическую (у слов в текстах указаны части речи и морфологические признаки);
        \2[\textbullet] семантическую разметку небольшого объёма\footnote{ 579 тыс. слов (60.3\% от общего количества слов в
         	             текстах) автоматически привязано к 1.5 млн  слов. 6 269 слов (1.1 \% от общего числа размеченных слов) 
         	             проверены вручную и подтверждены экспертом.
                        (По данным на 23 апреля 2020 г.)}, словa в текстах связаны со значениями словарных статей.
				  % http://dictorpus.krc.karelia.ru/ru/stats/by_corp
				  % select count(*) from meaning_text;                        
\end{outline}

Разрабатываемый корпус ВепКар можно назвать национальным корпусом 
вепсского и карельского языков\footnote{ Национальный корпус~-- это собрание текстов 
в электронной форме, 
отображающих данный язык во всём многообразии жанров, стилей, территориальных и официальных вариантов, 
тем самым представляя язык на определённом этапе своего существования~\cite[с.~418]{Kibrik2019}. 
Национальный корпус создаётся лингвистами для научных исследований и обучения языку~\cite[с.~419]{Kibrik2019}.
}.

Что такое лингвистический корпус текстов? Это, в первую очередь, компьютерная программа для обработки и анализа текстов или это сами тексты? 

Ответом на эти вопросы могут послужить слова Никлауса Вирта, что: 
<<...структура программы и строение данных неразрывно связаны между собой...>>~\cite[с.~9]{Wirth1989AlgorithmsAndDataStructure}. 
Более того:   
<<...данные предшествуют алгоритмам: нужно иметь некоторые объекты, 
прежде чем выполнять действия с ними...>>~\cite[с.~8]{Wirth1985Algorithms+} 
-- утверждается в книге <<Алгоритмы + структуры данных = программы>>.
Мы можем развить эту мысль применительно к лингвистическим корпусам:
        
\textWithVerticalBar{15cm}{Корпусный менеджер + тексты = корпус.}

\noindent
При этом корпусный менеджер обслуживает как сами тексты, так и словарные статьи. 
В соответствии с этим положением о важности данных, далее будет  
сначала описана база данных, затем корпусный менеджер и словарь.

\subsection{Приложения корпуса ВепКар в международных проектах} \label{sect_VepKar_international}


Отметим важность и востребованность работы по развитию корпуса текстов. 
Во-первых, в соревновании «Оценка методов обработки малоресурсных языков» (февраль-март, 2019), проводимом в рамках международной конференции «Диалог 2019», использовались размеченные данные корпуса ВепКар в формате CONLL (https://lowresource-lang-eval.github.io). 
Во-вторых, данные корпуса включены в этом (2019 или 2020?) году в международную морфологическую базу данных UniMorph (https://github.com/unimorph). Экспорт данных в общепринятые форматы (CONLL, UniMorph) важен для привлечения к исследованию вепсского и карельского языков международного научного сообщества. В последние годы становится практикой, когда новые методы и алгоритмы вычислительной лингвистики проверяются не на одном-двух языках, а на множестве языков, например, с привлечением базы UniMorph, включающей 110 языков, теперь, в том числе, вепсский и карельский. 

Новые данные вепсского и карельских словарей корпуса ВепКар экспортированы и включены в международный проект UniMorph 3.0 (международная морфологическая разметка):
* вепсский язык (https://github.com/unimorph/vep),
* собственно карельское наречие (https://github.com/unimorph/krl),
* карельский: ливвиковское наречие (https://github.com/unimorph/olo),
* карельский: людиковское наречие (https://github.com/unimorph/lud).

Был разработан специальный модуль для экспорта текстов в формате CONLL и 
для экспорта словарей в формате UD для проекта UniMorph \aka{Ссылки на статьи и проекты}.












