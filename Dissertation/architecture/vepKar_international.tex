\subsection{Приложения корпуса ВепКар в международных проектах} \label{sect_VepKar_international}


Отметим важность и востребованность работ по развитию корпуса текстов. 
Во-первых, в соревновании <<Оценка методов обработки малоресурсных языков>>, 
проведённом в феврале-марте 2019 года 
в рамках международной конференции <<Диалог>>, 
использовались размеченные данные корпуса ВепКар 
в формате CONLL~\cite{Klyachko2019LowresourceEval}. 
Результаты соревнования и данные корпусов 
доступны онлайн\footnote{См. \url{https://lowresource-lang-eval.github.io}}.

Во-вторых, данные корпуса с 2019 года включены 
в международную морфологическую базу данных 
UniMorph\footnote{См. \url{https://github.com/unimorph}}~\cite{McCarthy2020Unimorph}. 
Экспорт данных в общепринятые форматы (CONLL, UniMorph) 
важен для привлечения к исследованию вепсского и карельского языков 
международного научного сообщества. 
В последние годы становится практикой, 
когда новые методы и алгоритмы вычислительной лингвистики 
проверяются не на одном-двух языках, 
а на множестве языков, например, с привлечением базы UniMorph, включающей 110 языков. 
Таким образом, теперь в эту сотню языков входят вепсский язык и три диалекта карельского языка. 
\TODO{TODO: привести фрагмент таблицы с результатами вычислений по этим языкам (выборка из большой таблицы, см. Overleaf)}.

Новые данные вепсского и карельских словарей корпуса ВепКар экспортированы и включены в международный проект UniMorph 3.0 (международная морфологическая разметка):
* вепсский язык (https://github.com/unimorph/vep),
* собственно карельское наречие (https://github.com/unimorph/krl),
* карельский: ливвиковское наречие (https://github.com/unimorph/olo),
* карельский: людиковское наречие (https://github.com/unimorph/lud).

Был разработан специальный модуль для экспорта текстов в формате CONLL и 
для экспорта словарей в формате UD для проекта UniMorph \aka{Ссылки на статью и проект ~Tree Dependency Bank по ВепКар}.
 