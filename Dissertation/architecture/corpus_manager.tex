
\section{Архитектура корпусного менеджера Dictorpus} \label{sect_arch_corpus_manager}

Корпусный менеджер~--- система управления текстовыми и лингвистическими данными, 
специализированная поисковая система, включающая программные средства для поиска данных в корпусе, получения статистической информации и предоставления результатов пользователю 
в удобной форме~\cite[с.~3]{Zakharov2005}.

Кратко очертим функциональность корпусного менеджера Dictorpus, расширив список, предложенный
в работе~\cite[с.~10--11]{Zakharov2005}:
\begin{itemize}
    \item[+] поиск конкретных словоформ;
    \item[+] поиск словоформ по леммам;
    \item[--] поиск группы словоформ в виде разрывной или неразрывной синтагмы;
    \item[--] поиск словоформ по набору морфологических признаков;
    \item[+] отображение информации о происхождении, типе текста; 
        диалектные тексты сопровождаются подробной паспортизацией~\cite{Krizhanovsky2019Architecture};
    \item[+] вывод результатов поиска с указанием контекста заданной длины (в каждой словарной статье выводятся так называемые <<примеры>>, которые являются контекстами длиной в одно предложение);
    \item[?] получение различных лексико-грамматических статистических данных;
    \item[--] сохранение отобранных строк конкорданса в отдельном файле на компьютере пользователя и др.;
    \item[] \todo{Что ещё по-крупному есть в корпусе, по функциям?}
    \item[] 
    \item[] 
\end{itemize}
