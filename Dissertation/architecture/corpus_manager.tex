
\section{Архитектура корпусного менеджера Dictorpus} \label{sect_arch_corpus_manager}

Корпусный менеджер~--- система управления текстовыми и лингвистическими данными, 
специализированная поисковая система, включающая программные средства для поиска данных в корпусе, получения статистической информации и предоставления результатов пользователю 
в удобной форме~\cite[с.~3]{Zakharov2005}.

С~учётом возможностей, описанных в работах~\cite[с.~414--415]{Kibrik2019} и~\cite[с.~10--11]{Zakharov2005} 
функциональность корпусного менеджера Dictorpus выглядит следующим образом\footnote{Tе функции, 
которые ещё не реализованы в Dictorpus, отмечены знаком минус `--'.}:
\begin{itemize}
    \item[+] поиск слова (леммы) по любой его словоформе;
    \item[+] поиск словоформ по леммам;
    \item[--] поиск неразрывных и разрывных словосочетаний;
    \item[+] поиск словоформ по набору морфологических признаков;
    \item[+] отображение информации о происхождении, типе текста, авторе и прочем (\emph{метаразметка}); 
        диалектные тексты в ВепКар сопровождаются подробной паспортизацией~\cite{Krizhanovsky2019Architecture};
    \item[+] пользователь может создать свой подкорпус текстов, 
        отобранных по жанру, тематике, году записи \nata{Добавить поиск текстов по году записи и жанру.}, 
        указав параметры поиска, 
        то есть последующий поиск будет осуществляться именно в этом подкорпусе;
    
    \item[+] вывод результатов поиска в виде конкорданса\footnote{
            Конкорданс -- это множество контекстов, 
            в которых встретилось запрашиваемое языковое выражение~\cite[с.~415]{Kibrik2019}.
        }, при этом примеры\footnote{<<Примеры>> 
                                        в ВепКар являются контекстами 
                                        длиной в одно предложение.} 
                в ВепКар выводятся в словарной статье с информацией об источнике, откуда был взят пример; 
                также примеры сопровождаются сылкой на полный текст в корпусе;
    \item[+] получение различных лексико-грамматических статистических данных;
    \item[--] сохранение отобранных строк конкорданса в отдельном файле на компьютере пользователя;
    \item[--] вывод результатов поиска в текстах в формате ``Key Word in Context'' (KWIC);
    \item[--] можно получить статистическую информацию 
        о запрашиваемом языковом выражении (его относительную частоту по всему корпусу,
        распределение по жанрам или временным срезам, сочетаемость и частота сочетаемости с другими словами);
        
    \item[?] \nata{Todo: Какие важные функции в корпусе ещё добавить?}
\end{itemize}
