\chapter*{Словарь терминов}             % Заголовок
\addcontentsline{toc}{chapter}{Словарь терминов}  % Добавляем его в оглавление

\textbf{Компьютерная модель} "--- определение и построение взаимосвязи между входными данными (текстом) и результатом (лемма, морфологические свойства).

\textbf{Лемма} "--- базовая, каноническая форма слова.

\textbf{Морфология} "--- раздел лингвистики, который изучает структуру слова и его грамматические значения~\cite{MitreninaNikolaevLando2016}.

\textbf{Морфологический анализ} "--- определение леммы и его грамматических характеристик~\cite{MitreninaNikolaevLando2016}.

\textbf{Нормализация} "--- постановка слова или словосочетания в каноническую форму~\cite{MitreninaNikolaevLando2016}.

\textbf{Словоформа} "--- слово в определенной грамматической форме~\cite{MitreninaNikolaevLando2016}.

\textbf{Токенизация} "--- разбитие текста на предложения, а в каждом предложении выделение слов, знаков препинания и других элементов текста - числа, формулы, таблицы и т.д.~\cite{MitreninaNikolaevLando2016}.

\textbf{Токены} "--- выделенные в результате токенизации единицы (слова, числа, знаки препинания и пр.)~\cite{MitreninaNikolaevLando2016}.