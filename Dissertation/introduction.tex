\chapter*{Введение}                         % Заголовок
\addcontentsline{toc}{chapter}{Введение}    % Добавляем его в оглавление

\newcommand{\actuality}{}
\newcommand{\progress}{}
\newcommand{\aim}{{\textbf\aimTXT}}
\newcommand{\tasks}{\textbf{\tasksTXT}}
\newcommand{\novelty}{\textbf{\noveltyTXT}}
\newcommand{\influence}{\textbf{\influenceTXT}}
\newcommand{\methods}{\textbf{\methodsTXT}}
\newcommand{\defpositions}{\textbf{\defpositionsTXT}}
\newcommand{\reliability}{\textbf{\reliabilityTXT}}
\newcommand{\probation}{\textbf{\probationTXT}}
\newcommand{\contribution}{\textbf{\contributionTXT}}
\newcommand{\publications}{\textbf{\publicationsTXT}}


{\actuality} Обзор, введение в тему, обозначение места данной работы в
мировых исследованиях и~т.\:п., можно использовать ссылки на~другие
работы\ifnumequal{\value{bibliosel}}{1}{~\autocite{Gosele1999161}}{}
(если их~нет, то~в~автореферате
автоматически пропадёт раздел <<Список литературы>>). Внимание! Ссылки
на~другие работы в разделе общей характеристики работы можно
использовать только при использовании \verb!biblatex! (из-за технических
ограничений \verb!bibtex8!. Это связано с тем, что одна
и~та~же~характеристика используются и~в~тексте диссертации, и в
автореферате. В~последнем, согласно ГОСТ, должен присутствовать список
работ автора по~теме диссертации, а~\verb!bibtex8! не~умеет выводить в одном
файле два списка литературы).
При использовании \verb!biblatex! возможно использование исключительно
в~автореферате подстрочных ссылок
для других работ командой \verb!\autocite!, а~также цитирование
собственных работ командой \verb!\cite!. Для этого в~файле
\verb!Synopsis/setup.tex! необходимо присвоить положительное значение
счётчику \verb!\setcounter{usefootcite}{1}!.

Для генерации содержимого титульного листа автореферата, диссертации
и~презентации используются данные из файла \verb!common/data.tex!. Если,
например, вы меняете название диссертации, то оно автоматически
появится в~итоговых файлах после очередного запуска \LaTeX. Согласно
ГОСТ 7.0.11-2011 <<5.1.1 Титульный лист является первой страницей
диссертации, служит источником информации, необходимой для обработки и
поиска документа>>. Наличие логотипа организации на~титульном листе
упрощает обработку и~поиск, для этого разметите логотип вашей
организации в папке images в~формате PDF (лучше найти его в векторном
варианте, чтобы он хорошо смотрелся при печати) под именем
\verb!logo.pdf!. Настроить размер изображения с логотипом можно
в~соответствующих местах файлов \verb!title.tex!  отдельно для
диссертации и автореферата. Если вам логотип не~нужен, то просто
удалите файл с~логотипом.

\ifsynopsis
Этот абзац появляется только в~автореферате.
Для формирования блоков, которые будут обрабатываться только в~автореферате,
заведена проверка условия \verb!\!\verb!ifsynopsis!.
Значение условия задаётся в~основном файле документа (\verb!synopsis.tex! для
автореферата).
\else
Этот абзац появляется только в~диссертации.
Через проверку условия \verb!\!\verb!ifsynopsis!, задаваемого в~основном файле
документа (\verb!dissertation.tex! для диссертации), можно сделать новую
команду, обеспечивающую появление цитаты в~диссертации, но~не~в~автореферате.
\fi

% {\progress} 
% Этот раздел должен быть отдельным структурным элементом по
% ГОСТ, но он, как правило, включается в описание актуальности
% темы. Нужен он отдельным структурынм элемементом или нет ---
% смотрите другие диссертации вашего совета, скорее всего не нужен.

{\aim} данной работы является \ldots

Для~достижения поставленной цели необходимо было решить следующие {\tasks}:
\begin{enumerate}
  \item Исследовать, разработать, вычислить и~т.\:д. и~т.\:п.
  \item Исследовать, разработать, вычислить и~т.\:д. и~т.\:п.
  \item Исследовать, разработать, вычислить и~т.\:д. и~т.\:п.
  \item Исследовать, разработать, вычислить и~т.\:д. и~т.\:п.
\end{enumerate}


{\novelty}
\begin{enumerate}
  \item Впервые \ldots
  \item Впервые \ldots
  \item Было выполнено оригинальное исследование \ldots
\end{enumerate}

{\influence} \ldots

{\methods} \ldots

{\defpositions}
\begin{enumerate}
  \item Первое положение
  \item Второе положение
  \item Третье положение
  \item Четвертое положение
\end{enumerate}
В папке Documents можно ознакомиться в решением совета из Томского ГУ
в~файле \verb+Def_positions.pdf+, где обоснованно даются рекомендации
по~формулировкам защищаемых положений. 

{\reliability} Достоверность научных положений, основных выводов и результатов диссертационной работы 
основывается на анализе работ в этой области, согласованности предложенных моделей 
\ldots \todo{(какие модели и алгоритмы были разработаны, перечислить)} \ldots 
с результатами экспериментов, полученных на основе разработанного комплекса компьютерных программ. 
Также достоверность подтверждается апробацией основных положений диссертации в печатных трудах 
и докладах на научных конференциях. 
Новизна технических решений подтверждается полученными свидетельствами на программы для ЭВМ. 
Результаты работы представлялись на международной конференции <<Корпусная лингвистика>> в 2017, 2019 гг.; 
международной конференции <<Диалог>> в 2019 г.;
международной конференции <<Электронная письменность народов Российской Федерации: опыт, проблемы и перспективы>> в 2019 г., 
\ldots
\ldots; всероссийской конференции <<>> в 2019 г., \ldots 
Получены свидетельства о регистрации программы для ЭВМ \textnumero~2019665163 от 20.11.2019 "New written Tver Karelian dialects wordform generator" и 
\ldots
(см. Приложение А.).




{\contribution} Автор принимал активное участие \ldots

%\publications\ Основные результаты по теме диссертации изложены в ХХ печатных изданиях~\cite{Sokolov,Gaidaenko,Lermontov,Management},
%Х из которых изданы в журналах, рекомендованных ВАК~\cite{Sokolov,Gaidaenko}, 
%ХХ --- в тезисах докладов~\cite{Lermontov,Management}.

\ifnumequal{\value{bibliosel}}{0}{% Встроенная реализация с загрузкой файла через движок bibtex8
    \publications\ Основные результаты по теме диссертации изложены в XX печатных изданиях, 
    X из которых изданы в журналах, рекомендованных ВАК, 
    X "--- в тезисах докладов.%
}{% Реализация пакетом biblatex через движок biber
%Сделана отдельная секция, чтобы не отображались в списке цитированных материалов
    \begin{refsection}[vak,papers,conf]% Подсчет и нумерация авторских работ. Засчитываются только те, которые были прописаны внутри \nocite{}.
        %Чтобы сменить порядок разделов в сгрупированном списке литературы необходимо перетасовать следующие три строчки, а также команды в разделе \newcommand*{\insertbiblioauthorgrouped} в файле biblio/biblatex.tex
        \printbibliography[heading=countauthorvak, env=countauthorvak, keyword=biblioauthorvak, section=1]%
        \printbibliography[heading=countauthorconf, env=countauthorconf, keyword=biblioauthorconf, section=1]%
        \printbibliography[heading=countauthornotvak, env=countauthornotvak, keyword=biblioauthornotvak, section=1]%
        \printbibliography[heading=countauthor, env=countauthor, keyword=biblioauthor, section=1]%
        \nocite{%Порядок перечисления в этом блоке определяет порядок вывода в списке публикаций автора
                vakbib1,vakbib2,%
                confbib1,confbib2,%
                bib1,bib2,%
        }%
        \publications\ Основные результаты по теме диссертации изложены в~\arabic{citeauthor}~печатных изданиях, 
        \arabic{citeauthorvak} из которых изданы в журналах, рекомендованных ВАК, 
        \arabic{citeauthorconf} "--- в~тезисах докладов.
    \end{refsection}
    \begin{refsection}[vak,papers,conf]%Блок, позволяющий отобрать из всех работ автора наиболее значимые, и только их вывести в автореферате, но считать в блоке выше общее число работ
        \printbibliography[heading=countauthorvak, env=countauthorvak, keyword=biblioauthorvak, section=2]%
        \printbibliography[heading=countauthornotvak, env=countauthornotvak, keyword=biblioauthornotvak, section=2]%
        \printbibliography[heading=countauthorconf, env=countauthorconf, keyword=biblioauthorconf, section=2]%
        \printbibliography[heading=countauthor, env=countauthor, keyword=biblioauthor, section=2]%
        \nocite{vakbib2}%vak
        \nocite{bib1}%notvak
        \nocite{confbib1}%conf
    \end{refsection}
}
При использовании пакета \verb!biblatex! для автоматического подсчёта
количества публикаций автора по теме диссертации, необходимо
их~здесь перечислить с использованием команды \verb!\nocite!.
 % Характеристика работы по структуре во введении и в автореферате не отличается (ГОСТ Р 7.0.11, пункты 5.3.1 и 9.2.1), потому её загружаем из одного и того же внешнего файла, предварительно задав форму выделения некоторым параметрам

\textbf{Объем и структура работы.} Диссертация состоит из~введения, трёх глав,
заключения и~двух приложений.
%% на случай ошибок оставляю исходный кусок на месте, закомментированным
%Полный объём диссертации составляет  \ref*{TotPages}~страницу
%с~\totalfigures{}~рисунками и~\totaltables{}~таблицами. Список литературы
%содержит \total{citenum}~наименований.
%
Полный объём диссертации составляет
\formbytotal{TotPages}{страниц}{у}{ы}{}, включая
\formbytotal{totalcount@figure}{рисун}{ок}{ка}{ков} и
\formbytotal{totalcount@table}{таблиц}{у}{ы}{}.   Список литературы содержит
\formbytotal{citenum}{наименован}{ие}{ия}{ий}.

Разбить по частям и главам текст:

Цель исследования -- разработка методов и алгоритмов построения компьютерной модели морфологического словаря с целью решения задачи лемматизации на примере карельского языка. Для упрощения работы будет рассматриваться только именное словоизменение, то есть я ограничусь карельскими существительными, прилагательными, числительными и местоимениями.
На важность поставленной проблемы указывает тот факт, что любая задача, связанная с обработкой текста, требует первичной обработки текста, которая включает лемматизацию, то есть приведение слов к начальной форме.

Для успешного развития машинных методов обработки языка жизненно необходимо наличие трех компонент: корпуса текстов, словаря и морфологического анализатора (лемматизатора). Для работы с текстами на русском языке есть корпуса (Национальный корпус русского языка, OpenCorpora), есть словари и лемматизаторы (MyStem, pymorphy2). Для карельского языка есть корпус и словарь ВепКар, но лемматизатора нет.

Для обработки текстов на карельском языке, для машинного перевода текстов с карельского и на карельский язык нужен морфологический анализатор. Таким образом, задача построения такого анализатора будет не только своевременна (поскольку у нас есть корпус ВепКар), но и будет серьезным научным достижением для прибалтийско-финского языкознания. На данный момент мне неизвестно о существовании доступного лемматизатора карельского языка.

Укажу на взаимосвязь задачи построения морфологического анализатора с другими задачами вычислительной лингвистики. То есть определю связанные задачи, которые покажут комплексность поставленной проблемы. Итак, в ходе исследовательской работы потребуется решение следующих задач:
\begin{itemize}
\item сегментация и нормализация текста,
\item определение частеречной принадлежности (POS tagging),
\item определение морфологических свойств слова,
\item определение леммы слова для словарных и для неизвестных слов,
\item создание тестовой коллекции для оценки работы морфологических парсеров карельского языка.
\end{itemize}
