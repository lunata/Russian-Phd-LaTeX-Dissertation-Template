% жирная черта слева (или справа) от утверждения (таблица)
% https://tex.stackexchange.com/a/256732/99685
%\newcommand\thickPipe{!{\vrule width 4pt}}
% arguments: text and width in cm
\newcommand\textWithVerticalBar[2]{%
   \begin{tabular}{!{\vrule width 4pt}p{#1}}%
    #2%
    \end{tabular}%
}


%%%%%%%%%%%%%%%%%%%%%%%%%%%%%%%%%%%%%%%%%%%%%%%%%%%%%%%%%%%%%%%%%%%%%%%%%%%%%%
%
% todonotes section
%
% see userpackages.tex
%     \usepackage{todonotes}

\makeatletter
\newcommand*\iftodonotes{\if@todonotes@disabled\expandafter\@secondoftwo\else\expandafter\@firstoftwo\fi}  % defines \iftodonotes{<true>}{<false>}, thanks to https://tex.stackexchange.com/questions/126559/conditional-based-on-packageoption
\makeatother
\newcommand{\noindentaftertodo}{\iftodonotes{\noindent}{}}
% Note that these macros accept optional arguments such as size=\small, bordercolor=red, and so on.  Capitalized versions are inline paragraphs instead of margin notes.
\newcommand{\fixme}[2][]{\todo[color=yellow,size=\scriptsize,fancyline,caption={},#1]{#2}} % to mark stuff that you know is missing or wrong when you write the text
\newcommand{\note}[4][]{\todo[author=#2,color=#3,size=\scriptsize,fancyline,caption={},#1]{#4}} % default note settings, used by macros below.

\newcommand{\nata}[2][]{\note[#1]{Nata}{blue!40}{#2}}
\newcommand{\Nata}[2][]{\nata[inline,#1]{#2}\noindentaftertodo}

\newcommand{\aka}[2][]{\note[#1]{AKA}{teal!40}{#2}}
\newcommand{\AKA}[2][]{\aka[inline,#1]{#2}\noindentaftertodo}

\reversemarginpar
\setlength{\marginparwidth}{2cm}
%
% eo todonotes section
%
%%%%%%%%%%%%%%%%%%%%%%%%%%%%%%%%%%%%%%%%%%%%%%%%%%%%%%%%%%%%%%%%%%%%%%%%%%%%%%
