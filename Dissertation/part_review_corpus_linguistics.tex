\section{Корпусная лингвистика} \label{sect_review_corpus_linguistics}

Корпусная лингвистика~--- это раздел компьютерной (прикладной) лингвистики, 
описывающий принципы и методы 
построения лингвистических корпусов (корпусов текстов)
и методы использования корпусных данных~\cite[с.~3]{Zakharov2005}, \cite[с.~407]{Kibrik2019}.

Научное значение корпусов заключается в том, что наличие корпуса обеспечивает воспроизводимость, 
возможность повторить эксперимент~\cite[с.~409]{Kibrik2019}. 
Трудность здесь может крыться в том, что <<живые>> корпусы, 
то есть те, над которыми продолжают работать исследователи, 
постоянно пополняются новыми текстами, увеличивается объём разметки. 
Этот рост корпуса может менять результаты эксперимента. 
Ситуацию здесь может спасти, во-первых, пресловутая <<сбалансированность>> корпуса (см. следующий раздел). 
Во-вторых, в достаточно больших корпусах, добавление новых данных будет небольшим 
относительно всего корпуса.\footnote{%
    %
    % Объём и в процентах новых слов в ВепКар 
    Приведём пример от противного, пример о добавлении новых данных в корпус ВепКар. 
    Были обработаны и преобразованы в машиночаемую форму словарные статьи 
    Сопоставительно-ономасиологического словаря диалектов карельского, вепсского, саамского языков (кратко, словарь СОСД)~\cite{SOSD2007}. 
    %одержащего 1500 понятий 
    %\TODO{TODO можно ли чуть подробнее о СОСД? Сколько типов связей и сколько связей он содержит? Сколько диалектов? Чем он грандиозен?} 
    До включения данных словаря СОСД система ВепКар содержала 35~098 лемм, 800 переводов.
    После обработки СОСД в ВепКар было включено 1425 понятий 
        (\TODO{одно понятие через значение связывает несколько десятков диалектов}), 
    20 тыс. лемм, связанных с понятиями, из них 16 тыс. новых (\TODO{то есть число лемм выросло на 46\% от 35 тыс.}); 
    130 тыс. переводов (связей между значениями лемм из разных языков, наречий). 
    При этом было создано 60 тыс. связей между значениями лемм и словами из текста, 
    50 тыс. слов в текстах получили новые связи. Но только 2 тыс. слов до этого не имели разметки вообще.
    Эти цифры говорят и о том, что Сопоставительно-ономасиологический словарь был грандиозным проектом, 
    содержащим результаты работы большого коллектива учёных, и о том,
    что корпус ВепКар ещё далёк от насыщения по числу слов и текстов.
}



\subsection{Сбалансированность корпуса и достоверность данных}

В работе~\cite{Belikov2013}
оценивается достоверность корпусных исследований. 
Часто исследователи подвержены соблазну распространить опыт, полученный на конкретном корпусе, 
на весь язык, что неправомерно~\cite{Belikov2013}.

Каковы границы применимости разрабатываемого Открытого корпуса вепсского и карельского языков? 
Для ответа на этот вопрос нужно определить, тексты каких жанров и в какой пропорции включены в корпус ВепКар. 
Объём корпуса ВепКар составляет соответственно 2~677 текстов и 960~553 слов в текстах\footnote{ Данные на 23 апреля 2020~г. Cм. подробнее 
\href{http://dictorpus.krc.karelia.ru/ru/stats/by\_corp}{http://dictorpus.krc.karelia.ru/ru/stats/by\_corp}.}.

На рисунках \ref{fig:text_distr_by_corpus} и \ref{fig:text_distr_by_genre} показано распределение текстов по подкорпусам и жанрам для вепсского языка и наречий карельского языка\footnote{ Данные на 23 апреля 2020~г. Cм. подробнее 
			\href{http://dictorpus.krc.karelia.ru/ru/corpus/corpus}{http://dictorpus.krc.karelia.ru/ru/corpus/corpus}.}.

\begin{figure}
    \centering
    \includegraphics[width=1.0\textwidth,keepaspectratio=true]{text_distr_by_corpus.png}
    \caption{Распределение текстов по подкорпусам (вепсский язык и наречия карельского языка).}
    \label{fig:text_distr_by_corpus}
\end{figure}
%\bigskip

\begin{figure}
    \centering
    \includegraphics[width=1.0\textwidth,keepaspectratio=true]{text_distr_by_genre.png}
    \caption{Распределение текстов по жанрам (вепсский язык и наречия карельского языка).}
    \label{fig:text_distr_by_genre}
\end{figure}

Анализ временной динамики корпуса ВепКар показан на рисунке \ref{fig:text_distribution_by_date}. 2041 текстов (86,67\%) не имеют информации о дате записи.
\begin{figure}
    \centering
    \includegraphics[width=1.0\textwidth,keepaspectratio=true]{text_distribution_by_date.png}
    \caption{Распределение числа текстов по годам.}
    \label{fig:text_distribution_by_date}
\end{figure}

Своевременно ли говорить о сбалансированности и представительности корпуса ВепКар? Вопрос остаётся открытым, нужны дополнительные исследования.
%Ответом может послужить информация о доле слов из словаря, употреблённых в текстах корпуса. 
\nata{TODO: Подсчитать долю / процент слов (для каждого из языков), которые встречаются в текстах. Процент + абсолютное число слов словаря в текстах.}
(Добавить эту статистику в корпус?)


