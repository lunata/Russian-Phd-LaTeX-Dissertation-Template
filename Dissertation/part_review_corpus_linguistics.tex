\section{Корпусная лингвистика} \label{sect_review_corpus_linguistics}

Корпусная лингвистика --- это раздел лингвистики, изучающий ...~\cite{Zakharov2005}.



\subsection{Сбалансированность корпуса и достоверность данных}

В работе~\cite{Belikov2013}
оценивают достоверность корпусных исследований. 
Часто исследователи подвержены соблазну распространить опыт, полученный на конкретном корпусе, 
на весь язык, что неправомерно~\cite{Belikov2013}.

Каковы границы разрабатываемого Открытого корпуса вепсского и карельского языков? 
Для ответа на этот вопрос нужно определить, тексты каких жанров и в какой пропорции включены в корпус.

Todo: построить и добавить картинку со страницы:

http://dictorpus.krc.karelia.ru/ru/corpus/corpus


\subsection{Europarl Corpus and Words2Grids}

В работе~\cite{Fam2018tools} различают две структуры: \emph{Paradigm tables} 
и \emph{Analogical grids}. 

На языке Python разработана программа Words2Grids, которая по списку словоформ 
создаёт Analogical grids, которые, в свою очередь, нужны для создания 
таблиц склонений, то есть Paradigm tables.

Эксперименты проводились на 11 языках в корпусе текстов Europarl. 
Особняком стоят результаты по финскому (агглюнативному) языку (см. табл. 1 на с. 1063).


\subsection{Морфологическая разметка ГИКРЯ}

Читать и писать о работе~\cite{Selegey2016}...
