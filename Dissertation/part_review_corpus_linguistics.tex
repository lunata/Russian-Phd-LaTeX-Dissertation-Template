\section{Корпусная лингвистика} \label{sect_review_corpus_linguistics}

Корпусная лингвистика~--- это раздел компьютерной (прикладной) лингвистики, разрабатывающий принципы и методы  
построения лингвистических корпусов (корпусов текстов)
и методы использования корпусных данных~\cite[с.~3]{Zakharov2005}, \cite[с.~407]{Kibrik2019}.

Научное значение корпусов заключается в том, что наличие корпуса обеспечивает воспроизводимость, возможность повторить эксперимент~\cite[с.~409]{Kibrik2019}. Трудность здесь может крыться в том, что <<живые>> корпус\todo{ы}, то есть те над которыми продолжают работать исследователи, постоянно пополняются новыми текстами, увеличивается объём разметки. Ситуацию здесь может спасти пресловутая <<сбалансированность>> корпуса (см. следующий раздел). 
Во-вторых, в достаточно больших корпусах, добавление новых данных будет небольшим относительно всего корпуса. \todo{Todo: Объём и в процентах новых слов в ВепКар после обработки и извлечения данных из словарных статей Сопоставительно-ономасиологического словаря диалектов карельского, вепсского, саамского языков (кратко, словарь СОСД)~\cite{SOSD2007}}.
До включения данных словаря СОСД система ВепКар содержала 35~098 лемм.

\subsection{Сбалансированность корпуса и достоверность данных}

В работе~\cite{Belikov2013}
оценивается достоверность корпусных исследований. 
Часто исследователи подвержены соблазну распространить опыт, полученный на конкретном корпусе, 
на весь язык, что неправомерно~\cite{Belikov2013}.

Каковы границы применимости разрабатываемого Открытого корпуса вепсского и карельского языков? 
Для ответа на этот вопрос нужно определить, тексты каких жанров и в какой пропорции включены в корпус ВепКар. 
Объём корпуса ВепКар составляет соответственно 2~677 текстов и 960~553 слов в текстах\footnote{ Данные на 23 апреля 2020~г. Cм. подробнее 
\href{http://dictorpus.krc.karelia.ru/ru/stats/by\_corp}{http://dictorpus.krc.karelia.ru/ru/stats/by\_corp}.}.


\todo{Яндекс-диск/Nata/Диссертация/pix/Количество текстов по подкорпусам}
Рис. Количество текстов по подкорпусам
\todo{Todo: построить и добавить картинку со страницы:}\footnote{ Данные на 23 апреля 2020~г. Cм. подробнее 
\href{http://dictorpus.krc.karelia.ru/ru/corpus/corpus}{http://dictorpus.krc.karelia.ru/ru/corpus/corpus}.}

\bigskip

Анализ временной динамики корпуса ВепКар: 
% select `date`, count(*) from events where id in (select event_id from texts) group by `date` order by `date`;
% select count(*) from texts where event_id is null; 
\todo{Построить гистограмму с годами записи текстов, см. рис. 2: http://textualheritage.org/content/view/398/188/lang,russian/ }.

\todo{Указать долю текстов с неизвестной датой записи.}.

\bigskip

Своевременно ли говорить о сбалансированности и представительности корпуса ВепКар? 
Ответом может послужить информация о доле слов из словаря, употреблённых в текстах корпуса. 
\todo{TODO: Подсчитать долю / процент слов (для каждого из языков), которые встречаются в текстах. Процент + абсолютное число слов словаря в текстах.}
(Добавить эту статистику в корпус?)



\subsection{Europarl Corpus and Words2Grids}

В работе~\cite{Fam2018tools} различают две структуры: \emph{Paradigm tables} 
и \emph{Analogical grids}. 

На языке Python разработана программа Words2Grids, которая по списку словоформ 
создаёт Analogical grids, которые, в свою очередь, нужны для создания 
таблиц склонений, то есть Paradigm tables.

Эксперименты проводились на 11 языках в корпусе текстов Europarl. 
Особняком стоят результаты по финскому (агглюнативному) языку (см. табл. 1 на с. 1063).


\subsection{Морфологическая разметка ГИКРЯ}

Читать и писать о работе~\cite{Selegey2016}...
