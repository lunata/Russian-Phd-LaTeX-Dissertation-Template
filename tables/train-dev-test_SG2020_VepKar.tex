
\TODO{Перевести данные в табличку с размерами train, dev и test set для языков 
ВепКар на SIGMORPHON 2020
И на основе этого сделать вывод, какие из наших языков мало- или много- 
ресурсные.}

SIGMORPHON 2020 task 0

vep  train 94\,395 dev 13\,320 test 26\,422

krl  train 80\,216 dev 11\,225 test 22\,290

olo  train 43\,936 dev 6\,260 test 12\,515

lud  train 294 dev 41 test 82

\begin{table}
%\begin{adjustbox}{width=1\textwidth}
\small
\centering
\small
\begin{tabular}{c|r|r|r|r|r|r|>{\hspace{2em}}r|r|>{\hspace{2em}}r|r}
\toprule
%\multicolumn{1}{c}{\textbf{Lang}}&\multicolumn{3}{c}{\textbf{Total}}\\
\textbf{Язык / Наречие} & \textbf{Код языка} & \textbf{Train} & \textbf{Dev} & \textbf{Test} \\
%\cmidrule(lr){1-1} \cmidrule(lr){2-4} \cmidrule(lr){5-7} \cmidrule(lr){8-9} \cmidrule(lr){10-11}
% &Train& Dev & Test \\

\midrule
Карельский: собственно карельское наречие & krl&80216&11225&22290\\
Карельский: людиковское наречие & lud&294&41&82\\
Карельский: ливвиковское наречие & olo&43936&6260&12515\\
Вепсский & vep&94395&13320&26422\\
%\midrule
%est&26728&3820&7637&2.7&0.4&0.8&6.1&5.1&22.4&11.6\\
%fin&99403&14201&28401&0.0&0.0&0.0&0.0&0.0&32.6&17.2\\
%izh&763&112&224&0.0&0.0&0.0&0.0&0.0&42.9&22.3\\
%liv&2787&398&802&0.0&0.0&0.0&0.0&0.0&40.7&24.1\\
%\midrule
%ang&29270&4122&8197&11.8&1.8&3.4&21.6&21.9&35.1&21.3\\
%aze&5602&801&1601&11.9&1.9&4.0&22.3&20.9&31.5&20.2\\
%cre&4571&584&1174&18.5&2.1&4.9&29.8&29.6&5.5&2.7\\
%dan&17852&2550&5101&16.5&2.5&5.0&34.5&32.9&71.4&51.8\\
%nob&13263&1929&3830&10.5&1.8&3.1&18.5&19.7&80.5&70.5\\
%pei&10017&1349&2636&15.8&2.6&4.9&21.5&21.4&9.1&4.7\\
\bottomrule
\end{tabular}
%\end{adjustbox}
\caption{Number of samples in training, development, test sets, as well as statistics on systematic errors (inconsistency) and percentage of samples with lemmata observed in the training set.
(VepKar languages, other Uralic languaes and languages with the highest rates of inconsistency, see~\cite{Vylomova2020SIGMORPHON})
}
\label{tab:lang-stats}
\end{table}
